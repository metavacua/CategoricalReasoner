\chapter*{Logic Lattice: Extracted Conceptual Structure}
\addcontentsline{toc}{chapter}{Logic Lattice: Extracted Conceptual Structure}

\section*{Introduction}

This chapter presents the lattice of logical systems organized by structural rules and sequent restrictions, extracted from the deleted ontology content. The conceptual structure describes relationships between minimal, intuitionistic, dual-intuitionistic, classical, linear, affine, and relevant logics as a category where morphisms represent extensions and interpretations.

\section*{The Logic Lattice Structure}

\subsection*{LM (Monotonic Logic)}

LM is the minimal common sublogic for LJ and LDJ with the following characteristics:
\begin{itemize}
    \item No negation
    \item No explosion principle
    \item No Law of Excluded Middle (LEM)
    \item No Law of Non-Contradiction (LNC)
    \item Standard logical connectives: $\wedge$, $\vee$, $\top$, $\bot$
\end{itemize}

\subsection*{LJ (Intuitionistic Logic)}

LJ extends LM with:
\begin{itemize}
    \item Double Negation Introduction: $\Gamma, A \vdash \neg \neg A$
    \item Law of Non-Contradiction (LNC): $\neg(A \wedge \neg A)$
    \item Principle of Explosion (Ex Falso Quodlibet): $\bot \vdash A$
    \item Single-conclusion sequents (RHS restriction relative to LK, and LHS liberation relative to LM)
    \item Full structural rules: weakening, contraction, exchange on both sides
\end{itemize}

Wikidata reference: Q176786 (intuitionistic logic)

\subsection*{LDJ (Dual Intuitionistic Logic)}

LDJ extends LM with:
\begin{itemize}
    \item Double Negation Elimination: $\neg \neg A \vdash A, \Delta$
    \item Law of Excluded Middle (LEM): $\vdash A \vee \neg A$
    \item Single-antecedent sequents (LHS restriction relative to LK, RHS liberation relative to LM)
    \item Full structural rules
\end{itemize}

\subsection*{LK (Classical Logic)}

LK is the terminal logic in the lattice, validating:
\begin{itemize}
    \item Law of Excluded Middle (LEM): $A \vee \neg A$
    \item Law of Non-Contradiction (LNC): $\neg(A \wedge \neg A)$
    \item Principle of Explosion: $\bot \vdash A$
    \item Double Negation Elimination (DNE): $\Gamma, \neg\neg A \vdash A, \Delta$
    \item Double Negation Introduction: $\Gamma, A \vdash \neg \neg A$
    \item Multi-conclusion sequents (no restrictions)
    \item Full structural rules on both sides
\end{itemize}

Wikidata reference: Q236975 (classical logic)

\subsection*{LL (Linear Logic)}

Linear Logic is resource-sensitive:
\begin{itemize}
    \item No weakening (no erasure)
    \item No contraction (no cloning)
    \item Exchange only (order matters for resource management)
    \item Multi-conclusion sequents
    \item Embodies quantum-safe principles (no-cloning, no-erasure theorems)
\end{itemize}

Wikidata reference: Q841728 (linear logic)

\subsection*{ALL (Affine Linear Logic)}

ALL extends LL with:
\begin{itemize}
    \item Weakening allowed (erasure permitted)
    \item No contraction (still no cloning)
    \item Exchange
\end{itemize}

\subsection*{RLL (Relevant Linear Logic)}

RLL extends LL with:
\begin{itemize}
    \item Contraction allowed (cloning permitted)
    \item No weakening (no erasure)
    \item Exchange
\end{itemize}

\section*{Morphisms in the Logic Category}

\subsection*{Extension Morphisms}

Morphisms in the logic category represent relationships between logics:

\begin{itemize}
    \item \textbf{LM $\to$ LJ}: Extension by adding LNC and Explosion, liberating LHS
    \item \textbf{LM $\to$ LDJ}: Extension by adding LEM, liberating RHS
    \item \textbf{LJ $\to$ LK}: Add LEM, liberating RHS restriction
    \item \textbf{LDJ $\to$ LK}: Add LNC, liberating LHS restriction
    \item \textbf{LL $\to$ ALL}: Add weakening (allows erasure)
    \item \textbf{LL $\to$ RLL}: Add contraction (allows cloning)
    \item \textbf{ALL $\to$ LK}: Add contraction and classical axioms
    \item \textbf{RLL $\to$ LK}: Add weakening and classical axioms
\end{itemize}

\subsection*{Structural Rules}

\begin{description}
    \item[Weakening] Allows addition of arbitrary formulas. Absence implies ``no erasure'' principle.
    \item[Contraction] Allows duplication of formulas. Absence implies ``no cloning'' principle.
    \item[Exchange] Allows reordering of formulas. Absence implies order-sensitive (non-commutative) logic.
\end{description}

\begin{verbatim}
@misc{zizzi2010quantummetalanguagelogicqubits,
      title={From Quantum Metalanguage to the Logic of Qubits}, 
      author={Paola Zizzi},
      year={2010},
      eprint={1003.5976},
      archivePrefix={arXiv},
      primaryClass={quant-ph},
      url={https://arxiv.org/abs/1003.5976}, 
}

@article{Zang_2025,
   title={No-Go Theorems for Universal Entanglement Purification},
   volume={134},
   ISSN={1079-7114},
   url={http://dx.doi.org/10.1103/PhysRevLett.134.190803},
   DOI={10.1103/physrevlett.134.190803},
   number={19},
   journal={Physical Review Letters},
   publisher={American Physical Society (APS)},
   author={Zang, Allen and Chen, Xinan and Chitambar, Eric and Suchara, Martin and Zhong, Tian},
   year={2025},
   month=may }
\end{verbatim}
   
Exchange can be mapped to the no-go theorem for partial swapping of quantum information.

\begin{verbatim}
@misc{chakrabarty2007impossibilitypartialswappingquantum,
      title={Impossibility of partial swapping of Quantum Information}, 
      author={Indranil Chakrabarty},
      year={2007},
      eprint={quant-ph/0612123},
      archivePrefix={arXiv},
      primaryClass={quant-ph},
      url={https://arxiv.org/abs/quant-ph/0612123}, 
}
\end{verbatim}

\section*{Logical Axioms}

\begin{description}
    \item[LEM] $A \vee \neg A$ --- Law of Excluded Middle. Every proposition is either true or false.
    \item[LNC] $\neg(A \wedge \neg A)$ --- Law of Non-Contradiction. No proposition can be both true and false.
    \item[Explosion] $\bot \vdash A$ --- Ex Falso Quodlibet. From contradiction, anything follows.
    \item[DNE] $\neg\neg A \vdash A$ --- Double Negation Elimination. Classical principle, not intuitionistically valid.
\end{description}

\section*{Commutative Diagrams}

The logic lattice forms a category with commutative diagrams:

\begin{center}
\begin{tikzcd}
    & \text{LM} \arrow[dl] \arrow[dr] & \\
    \text{LJ} \arrow[dr] & & \text{LDJ} \arrow[dl] \\
    & \text{LK} &
\end{tikzcd}
\end{center}

\begin{center}
\begin{tikzcd}
    & \text{LL} \arrow[dl] \arrow[dr] & \\
    \text{ALL} \arrow[dr] & & \text{RLL} \arrow[dl] \\
    & \text{LK} &
\end{tikzcd}
\end{center}
