% =============================================================================
% AMS Theorem Environments
% =============================================================================

% Theorem styles
\theoremstyle{plain}
\newtheorem{theorem}{Theorem}[chapter]
\newtheorem{lemma}[theorem]{Lemma}
\newtheorem{corollary}[theorem]{Corollary}
\newtheorem{proposition}[theorem]{Proposition}

\theoremstyle{definition}
\newtheorem{definition}[theorem]{Definition}
\newtheorem{example}[theorem]{Example}
\newtheorem{exercise}[theorem]{Exercise}
\newtheorem{axiom}{Axiom}[chapter]

\theoremstyle{remark}
\newtheorem{remark}[theorem]{Remark}
\newtheorem{notation}[theorem]{Notation}
\newtheorem{convention}[theorem]{Convention}

% Proof environment (already defined in amsthm, but customize here if needed)
\renewcommand{\qedsymbol}{$\blacksquare$}

% =============================================================================
% Core AMS and Math Packages
% =============================================================================

\usepackage{amsmath}
\usepackage{mathtools}
\usepackage{amsthm}
\usepackage{amssymb}

% =============================================================================
% Proof Theory and Logic
% =============================================================================

% Proof trees (ebproof is more modern, bussproofs is classic)
\usepackage{ebproof}
% Alternative: \usepackage{bussproofs}

% Sequent calculus notation
\usepackage{turnstile}

% Denotational semantics symbols
\usepackage{stmaryrd}

% =============================================================================
% Category Theory
% =============================================================================

% Commutative diagrams
\usepackage{tikz-cd}

% =============================================================================
% Additional Math Support
% =============================================================================

% Extended math symbols
\usepackage{mathrsfs}

% Better fractions
\usepackage{xfrac}

% =============================================================================
% Document Formatting
% =============================================================================

% Better tables
\usepackage{booktabs}
\usepackage{array}

% Code listings
\usepackage{listings}
\lstset{
  basicstyle=\ttfamily\small,
  breaklines=true,
  columns=fullflexible,
  frame=single,
  backgroundcolor=\color{gray!10},
}

% Hyperlinks
\usepackage[hidelinks]{hyperref}

% Colors
\usepackage{xcolor}

% =============================================================================
% Custom Commands
% =============================================================================

% Common logical connectives
\newcommand{\limp}{\rightarrow}           % Implication
\newcommand{\liff}{\leftrightarrow}       % Biconditional
\newcommand{\landop}{\wedge}              % Conjunction
\newcommand{\lorop}{\vee}                 % Disjunction
\newcommand{\lnotop}{\neg}                % Negation

% Sequent calculus
\newcommand{\seq}{\vdash}                 % Turnstile
\newcommand{\dseq}{\Vdash}                # Double turnstile
\newcommand{\lseq}{\dashv}                % Left turnstile

% Linear logic
\newcommand{\loll}{\multimap}             % Lollipop (linear implication)
\newcommand{\tensor}{\otimes}             % Tensor product
\newcommand{\parr}{\par}                  % Par (multiplicative disjunction)
\newcommand{\with}{\&}                    % With (additive conjunction)
\newcommand{\oplus}{\oplus}               % Oplus (additive disjunction)
\newcommand{\bang}{!}                     % Of course (exponential)
\newcommand{\whynot}{?}                   % Why not (exponential)

% Category theory
\newcommand{\cat}[1]{\mathbf{#1}}          % Category name
\newcommand{\op}{\mathrm{op}}             % Opposite category
\newcommand{\homf}{\mathrm{Hom}}          % Hom functor
\newcommand{\nat}{\Rightarrow}            % Natural transformation

% Common categories
\newcommand{\Set}{\cat{Set}}
\newcommand{\Cat}{\cat{Cat}}
\newcommand{\Pos}{\cat{Pos}}
\newcommand{\Mon}{\cat{Mon}}
\newcommand{\Grp}{\cat{Grp}}

% Code
\newcommand{\code}[1]{\texttt{#1}}

% =============================================================================
% QED Symbol Placement
% =============================================================================

% Ensure QED symbol appears at end of proofs, even with equations
\renewenvironment{proof}[1][\proofname]{%
  \par\pushQED{\qed}%
  \normalfont \topsep6\p@\@plus6\p@\relax
  \trivlist\item[\hskip\labelsep\textit{#1}.]%
  \ignorespaces
}{%
  \popQED\endtrivlist\@endpefalse
}
