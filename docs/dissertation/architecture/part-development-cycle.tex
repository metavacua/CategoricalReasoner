\chapter*{The Software Development Lifecycle}
\addcontentsline{toc}{chapter}{The Software Development Lifecycle}

The development lifecycle of the \catty\ framework is a deterministic process designed to eliminate ambiguity and ensure that every artifact meets formal acceptance criteria.

\section*{Development Workflow}
A standard development cycle for a single task follows an eight-step protocol:
\begin{enumerate}
    \item \textbf{Ingestion}: Read the task specification from \code{operations.yaml}.
    \item \textbf{Verification}: Ensure all required dependencies exist and are valid.
    \item \textbf{Specification}: Read the target artifact specification (path, format, content spec, schema).
    \item \textbf{Execution}: Follow the operational instructions in the task description exactly.
    \item \textbf{Content Validation}: Verify the artifact satisfies its \code{content\_spec}.
    \item \textbf{Criteria Testing}: Evaluate the boolean acceptance criteria.
    \item \textbf{Formal Validation}: Run the automated \code{validate.py} script (syntax, RDF, SHACL).
    \item \textbf{Completion}: Mark the task as complete and proceed to the next node in the dependency graph.
\end{enumerate}

\section*{Acceptance Criteria Philosophy}
Every task in the framework must have testable boolean acceptance criteria. The framework explicitly forbids aspirational or subjective criteria (e.g., "high quality", "comprehensive"). Instead, criteria must be binary and verifiable:
\begin{itemize}
    \item $\checkmark$ \textit{Testable}: "File \code{src/ontology/logics-as-objects.jsonld} exists."
    \item $\checkmark$ \textit{Testable}: "Contains at least 7 logic instances."
    \item $\checkmark$ \textit{Testable}: "Validates against \code{logics-as-objects.shacl}."
\end{itemize}

\section*{Continuous Validation}
Validation is not a final step but a continuous process. Artifacts are validated immediately upon creation or modification. The \GGG\ integration means that semantic web data is validated against live Wikidata endpoints to prevent identifier decay.

\section*{Licensing and Open Source}
All software components of the \catty\ framework are licensed under \textbf{AGPLv3}, while documentation and thesis content are licensed under \textbf{CC BY-SA 4.0}. This ensures that both the categorical theoretical foundations and their software implementations remain open and reproducible.
