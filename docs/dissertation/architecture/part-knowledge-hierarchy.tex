\chapter*{Knowledge Hierarchy}
\addcontentsline{toc}{chapter}{Knowledge Hierarchy}

The \catty\ framework adopts a strict hierarchy for knowledge sourcing to ensure semantic integrity and avoid the common pitfalls of LLM-based content generation.

\section*{Sourcing Priority Chain}
Knowledge is consumed and validated according to the following priority chain. We prioritize in each step open source and open science alternatives over any closed source or proprietary sources:
\begin{enumerate}
    \item \textbf{Canonical systems}: Primary expressions, representations, and code.
    \item \textbf{International standards}: ISO/IEC, IETF, and other formal global specifications.
    \item \textbf{Community standards}: De facto standards and community-driven specifications (e.g., Wikidata, DBPedia).
    \item \textbf{Original Research}: Specific innovations and novel contributions of the Catty thesis.
\end{enumerate}

\section*{Wikidata Discovery Protocol}
To prevent the corruption of the semantic knowledge graph with hallucinated identifiers, agents must follow the dynamic discovery protocol.

\subsection*{The Problem: Hallucinated QIDs}
LLMs often produce plausible-looking but incorrect Wikidata QIDs. For example, claiming "Classical Logic" is \code{Q192960} (which actually refers to Baibars). 

\textbf{Mandatory Rule}: Never guess a QID. All QIDs must be verified against the live Wikidata endpoint before use in the repository.

\subsection*{Verification and Discovery}
To verify a QID, use a SPARQL query. Note: When executing, extract ONLY the SPARQL code, excluding any LaTeX environment markers:
\begin{lstlisting}[language=SPARQL]
SELECT ?label ?description WHERE {
  wd:Q193138 rdfs:label ?label .
  wd:Q193138 schema:description ?description .
  FILTER(LANG(?label) = "en")
}
\end{lstlisting}

If a QID is unknown, discover it using a more efficient label-based query. Note that Wikidata labels for mathematical concepts are typically lowercase:
\begin{lstlisting}[language=SPARQL]
SELECT DISTINCT ?item ?label ?description WHERE {
  ?item rdfs:label "natural transformation"@en .
  ?item schema:description ?description .
  FILTER(LANG(?description) = "en")
}
LIMIT 5
\end{lstlisting}

\section*{Core Domain Registry}
The following verified identifiers constitute the core of the \catty\ domain model:

\begin{table}[h]
\centering
\begin{tabular}{@{}lll@{}}
\toprule
Concept & QID & Description \\ \midrule
Category Theory & \code{wd:Q217413} & Branch of mathematics \\
Category & \code{wd:Q719395} & Algebraic structure \\
Functor & \code{wd:Q864475} & Mapping between categories \\
Logic & \code{wd:Q8078} & Study of correct reasoning \\
Sequent Calculus & \code{wd:Q1771121} & Formal argumentation style \\
\bottomrule
\end{tabular}
\caption{Core Wikidata Identifiers}
\end{table}

\section*{SPARQL Query Testing Methodology}

To ensure integrity and reproducibility, all SPARQL queries documented in this thesis must be manually tested against live endpoints. The testing protocol follows strict validation criteria to prevent LLM-generated artifacts.

\subsection*{Testing Protocol}

All SPARQL queries are executed using \code{curl} with the following constraints:

\begin{enumerate}
    \item \textbf{Actual Endpoint Execution}: All queries must be executed against live SPARQL endpoints (Wikidata, DBPedia, etc.) using HTTP POST requests.
    \item \textbf{Zero LLM Generation}: Result files (JSON, XML, TTL) must be actual endpoint outputs, with zero tolerance for LLM-generated content.
    \item \textbf{Timeout Threshold}: Queries must complete within 60 seconds.
    \item \textbf{Non-Empty Results}: Valid queries must return at least one binding.
    \item \textbf{Reproducibility}: All curl commands and result files are preserved for verification.
\end{enumerate}

\subsection*{Query Classification}

Queries are classified according to their executability status:

\begin{itemize}
    \item \textbf{VALID}: Executes successfully against external endpoint, returns well-formed results in under 60s
    \item \textbf{INVALID-SYNTAX}: Contains SPARQL syntax errors or endpoint errors
    \item \textbf{INVALID-TIMEOUT}: Execution exceeds 60-second threshold
    \item \textbf{INVALID-EMPTY}: Returns valid format but zero results
    \item \textbf{NOT EXECUTABLE}: Requires local ontology not available as external endpoint
\end{itemize}

\subsection*{Tested Queries}

Three external Wikidata queries have been manually tested and validated:

\begin{table}[h]
\centering
\begin{tabular}{@{}llll@{}}
\toprule
Query & Purpose & Execution Time & Status \\ \midrule
Q1 & QID Verification & 0.754s & VALID \\
Q2 & Label Discovery & 0.367s & VALID \\
Q3 & Logic Instances & 0.939s & VALID \\
\bottomrule
\end{tabular}
\caption{Tested External SPARQL Queries}
\end{table}

All query results are preserved in \code{docs/dissertation/sparql-results/} with complete curl output and timing information. Full query listings and results are provided in Appendix A.

\subsection*{Local Catty Ontology Queries}

The 13 SPARQL queries in \code{docs/sparql-examples.md} target a local Catty categorical ontology that does not exist as an external SPARQL endpoint. These queries are classified as \textbf{NOT EXECUTABLE} and serve as specification documents for future ontology deployment. They define the intended structure of the Catty ontology including:

\begin{itemize}
    \item Logic instances (\code{catty:Logic})
    \item Morphisms between logics (\code{catty:Extension})
    \item Adjoint functor relationships (\code{catty:AdjointFunctors})
    \item Curry-Howard correspondences (\code{catty:correspondsToLogic})
\end{itemize}

These queries are preserved as reference examples in the thesis appendices.
