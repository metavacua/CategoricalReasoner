\chapter*{Architecture --- The Lattice of Logics}
\addcontentsline{toc}{chapter}{Architecture --- The Lattice of Logics}

\section*{The Two-Dimensional Lattice Structure}
The lattice is organized along two primary axes that define the logic-space:
\begin{itemize}
    \item \textbf{Horizontal Axis}: Represents sequent restrictions. It is symmetric at the origin, with single-succedent restrictions on the right and single-antecedent restrictions on the left.
    \item \textbf{Vertical Axis}: Represents structural rule configurations (weakening, contraction, exchange).
\end{itemize}
The 2D lattice serves as a basis for the category that constitutes the thesis.

\section*{Initial Logics}
Catty utilizes "initial logics" as the language for categorizable logics, avoiding namespace clashes with existing "Minimal" or "Basic" logics. Initial logics are relative to the category in question.

LM serves as an initial common sublogic for variants such as LJ and LDJ. It is related to the work of Sambin \cite{sambin2000basic} as a reduction or fragment of Basic Logic.

LL serves as an initial object in substructural contexts, containing three sub-linear logics that reflect linearized versions of foundational basis logics (LM, LJ, and LDJ).

\section*{Terminal Objects and Closure}
LK is the terminal logic in the category of subclassical logics. Every logic in the category extends toward classical closure.

\section*{Geometric and Cubic Structures}
The basis of the category can be represented as a cubic structure or polytope, where each dimension represents the inclusion of symmetric structural rules. This can be complexified by examining the asymmetry between left and right structural rules (e.g., LJ vs LDJ).
