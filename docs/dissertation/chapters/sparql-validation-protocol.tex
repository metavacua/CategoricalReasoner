\chapter{SPARQL Validation Protocol}
\label{sec:sparql-validation-protocol}

\section{Introduction}

This chapter documents the logical validity framework for SPARQL query execution in the context of the Catty thesis. When querying external semantic web endpoints, we must distinguish between three distinct outcomes:

\begin{enumerate}
  \item \textbf{Empty results + successful execution}: A well-formed query executes without error but returns zero results. This constitutes a \textit{proof of negative}---demonstrating that no matching data exists in the queried endpoint's knowledge graph.
  \item \textbf{Runtime errors / compilation failures}: A query fails to execute due to syntax errors, type mismatches, or endpoint issues. This indicates a \textit{malformed query} and is considered inconclusive.
  \item \textbf{Non-empty results + successful execution}: A well-formed query executes without error and returns one or more results. This constitutes a \textit{constructive witness}---demonstrating the existence of matching data in the queried endpoint's knowledge graph.
\end{enumerate}

This framework aligns with model-theoretic semantics for query languages \cite{maclane1971categories}, where query execution provides evidence about the structure of the underlying model.

\section{Logical Validity Framework}

\subsection{Model-Theoretic Interpretation}

From a model-theoretic perspective, a SPARQL endpoint can be viewed as an interpretation function $\mathcal{I}$ mapping RDF triples to truth values. A query $Q$ can be interpreted as a formula $\phi_Q$ over this interpretation. The execution of $Q$ against endpoint $E$ yields:

\begin{itemize}
  \item $\mathcal{I}_E \models \phi_Q$ (satisfied): Non-empty results (constructive witness)
  \item $\mathcal{I}_E \not\models \phi_Q$ (not satisfied): Empty results (proof of negative)
  \item $\phi_Q$ is ill-formed: Compilation/execution error (malformed query)
\end{itemize}

This interpretation respects the Curry-Howard correspondence \cite{curyhoward1934, howard1969formulae}, where query results serve as witnesses for the existence of certain structures in the knowledge graph.

\subsection{Outcome Categories}

\subsubsection{Proof of Negative}

When a well-formed SPARQL query executes successfully but returns zero results, this provides constructive evidence that the queried endpoint does not contain data matching the query pattern. This is fundamentally different from a failed execution.

\textit{Example}: Querying Wikidata for instances of a non-existent class using valid SPARQL syntax returns empty results, proving that no such instances exist in Wikidata's knowledge graph.

\subsubsection{Malformed Query}

A query that fails to compile or execute due to syntax errors, prefix declarations, or type mismatches provides no information about the underlying data model. Such queries must be corrected before they can serve as valid queries.

\textit{Example}: A SPARQL query with an undefined prefix or malformed triple pattern fails during parsing, indicating a query formulation problem rather than a data availability issue.

\subsubsection{Constructive Witness}

A well-formed query that executes successfully and returns non-empty results provides direct evidence of the existence of matching structures in the queried endpoint. These results can be treated as witnesses \cite{negri2011proof} for the query's satisfaction.

\textit{Example}: Querying Wikidata for logic-related entities returns a list of QIDs and labels, demonstrating that such entities exist and are structured according to Wikidata's ontology.

\section{Implementation}

\subsection{Query Execution Protocol}

All SPARQL queries in this project are executed against external endpoints following a strict protocol:

\begin{enumerate}
  \item Query formulation using well-defined prefixes (e.g., \texttt{wd:}, \texttt{dbo:}) from authoritative namespaces
  \item Syntax validation ensuring well-formed SPARQL
  \item Execution with appropriate HTTP headers (User-Agent, Accept)
  \item Capture of execution time and result count
  \item Validation of result format (TTL for CONSTRUCT, JSON for SELECT)
\end{enumerate}

\subsection{Valid Query Examples}

\subsubsection{Wikidata Logics Query}

The following query extracts logic-related entities from Wikidata:

\begin{verbatim}
PREFIX wd: <http://www.wikidata.org/entity/>
PREFIX wdt: <http://www.wikidata.org/prop/direct/>
PREFIX rdfs: <http://www.w3.org/2000/01/rdf-schema#>

# Extract logic-related entities from Wikidata
CONSTRUCT {
  ?logic rdfs:label ?label ;
         wdt:P31 wd:Q8078 .
}
WHERE {
  ?logic wdt:P31 wd:Q8078 .
  ?logic rdfs:label ?label .
  FILTER(LANG(?label) = "en")
}
LIMIT 10
\end{verbatim}

This query uses the Wikidata namespace (authority: \texttt{wikidata.org}) and queries for instances of "logic" (QID: Q8078). When executed, it provides a constructive witness of logic-related entities in Wikidata.

\subsubsection{DBPedia Category Theory Query}

The following query extracts category theory concepts from DBPedia:

\begin{verbatim}
PREFIX dbo: <http://dbpedia.org/ontology/>
PREFIX rdfs: <http://www.w3.org/2000/01/rdf-schema#>
PREFIX rdf: <http://www.w3.org/1999/02/22-rdf-syntax-ns#>
PREFIX dct: <http://purl.org/dc/terms/>

# Extract categories related to Category Theory from DBPedia
CONSTRUCT {
  ?concept rdfs:label ?label ;
           rdf:type dbo:MathematicalConcept .
}
WHERE {
  ?concept rdfs:label ?label .
  ?concept dct:subject <http://dbpedia.org/resource/Category:Category_theory> .
  FILTER(LANG(?label) = "en")
}
LIMIT 10
\end{verbatim}

This query uses the DBPedia ontology namespace (authority: \texttt{dbpedia.org}) and queries for concepts classified under category theory. When executed, it provides a constructive witness of category theory resources in DBPedia.

\section{Invalid Query Patterns}

\subsection{Local Ontology Prefixes}

Queries using prefixes like \texttt{catty:} that reference non-existent local ontologies are invalid for external endpoint queries. Such prefixes may be used only with local RDF files, and only when those files exist.

\textit{Invalid Example}:
\begin{verbatim}
PREFIX catty: <https://github.com/metavacua/CategoricalReasoner/ontology/>

SELECT ?logic ?label
WHERE {
  ?logic a catty:Logic ;
         rdfs:label ?label .
}
\end{verbatim}

This query attempts to use a \texttt{catty:} prefix against external endpoints where no such ontology exists. It may return empty results (proof of negative) or fail depending on endpoint behavior, but it serves no valid purpose against external sources.

\subsection{Execution Evidence}

All documented queries must be actually executed against external endpoints, with execution evidence preserved as valid TTL (for CONSTRUCT queries) or CSV (for SELECT queries). Fabricating results or using internal knowledge to simulate query output is strictly prohibited.

\section{Query Quality Requirements}

Well-formed SPARQL queries in this project must satisfy the following quality criteria:

\begin{enumerate}
  \item \textbf{Syntax Validity}: Query must parse and compile without errors
  \item \textbf{Prefix Authority}: All prefixes must reference authoritative namespaces (e.g., wikidata.org, dbpedia.org, w3.org)
  \item \textbf{Execution Success}: Query must execute without runtime errors
  \item \textbf{Result Format}: Results must be in valid TTL or JSON format
  \item \textbf{Timeout Compliance}: Query execution must complete within 60 seconds
\end{enumerate}

Queries that satisfy these criteria but return empty results are considered valid (proof of negative). Queries that fail any criterion are considered malformed and must be corrected.

\section{Demonstration Results}

\subsection{Wikidata Execution Results}

Execution of the Wikidata logics query against \texttt{https://query.wikidata.org/sparql} typically returns 10 logic-related entities with QIDs and labels. This serves as a constructive witness that Wikidata contains structured data about formal logics.

Example result structure:
\begin{verbatim}
<http://www.wikidata.org/entity/Q11448> a <http://www.wikidata.org/prop/direct/P31> ;
    <http://www.w3.org/2000/01/rdf-schema#label> "logic"@en .
\end{verbatim}

\subsection{DBPedia Execution Results}

Execution of the DBPedia category theory query against \texttt{https://dbpedia.org/sparql} typically returns 10 mathematical concepts with labels and classifications. This serves as a constructive witness that DBPedia contains structured data about category theory.

Example result structure:
\begin{verbatim}
<http://dbpedia.org/resource/Category_theory> a <http://dbpedia.org/ontology/MathematicalConcept> ;
    <http://www.w3.org/2000/01/rdf-schema#label> "Category theory"@en .
\end{verbatim}

\section{Integration with Thesis Development}

The SPARQL validation protocol directly supports thesis development by:

\begin{itemize}
  \item Providing evidence-based citations from external semantic web sources
  \item Enabling verification of QID and URI references used in thesis content
  \item Supporting the extraction of authoritative definitions and classifications
  \item Facilitating the discovery of related work through structured queries
\end{itemize}

All SPARQL-derived content in the thesis must be traceable to actual query executions with preserved evidence, maintaining the thesis's commitment to rigorous methodology \cite{lawvere1963functorial}.
