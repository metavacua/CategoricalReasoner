% Theorem: Contraction and Associative Resource Management
% ID: thm-contraction-associativity
% Research Source: DBpedia - Associativity isomorphism, Structural proof theory

% Minimal preamble for standalone theorem/proof documents
% This preamble is designed for compiling individual theorems

\documentclass[12pt]{article}

% Essential packages
\usepackage{amsmath}
\usepackage{amssymb}
\usepackage{amsthm}

% Theorem environments
\theoremstyle{definition}
\newtheorem{theorem}{Theorem}[section]
\newtheorem{lemma}[theorem]{Lemma}
\newtheorem{corollary}[theorem]{Corollary}
\newtheorem{proposition}[theorem]{Proposition}
\newtheorem{definition}[theorem]{Definition}

\theoremstyle{remark}
\newtheorem{remark}{Remark}

% Proof environment
\usepackage{proof}

% Sequent notation
\newcommand{\sequent}[2]{#1 \vdash #2}
\newcommand{\Sequent}{\vdash}

% Focused sequent calculus notation
\newcommand{\uparrow}{\mathbin{\uparrow}}
\newcommand{\downarrow}{\mathbin{\downarrow}}
\newcommand{\foc}[2]{#1 \mathbin{\Downarrow} #2}  % Focused sequent
\newcommand\focal{
  \mathchoice{\mathbin{\Downarrow}}{\mathbin{\Downarrow}}{\mathbin{\scriptstyle\Downarrow}}{\mathbin{\scriptstyle\Downarrow}}
}
\newcommand{\async}[2]{#1 \uparrow #2}  % Asynchronous phase
\newcommand{\sync}[2]{#1 \downarrow #2}  % Synchronous phase

% Structural rules
\newcommand{\weakening}{\textsf{W}}
\newcommand{\contraction}{\textsf{C}}
\newcommand{\exchange}{\textsf{E}}

% Focused structural rules
\newcommand{\fweakening}{\textsf{W}^{\downarrow}}
\newcommand{\fcontraction}{\textsf{C}^{\downarrow}}
\newcommand{\fexchange}{\textsf{E}^{\downarrow}}

% Inference rules
\newcommand{\infrule}[3]{\dfrac{#1}{#2}\;\;{\scriptsize \textsc{#3}}}
\newcommand{\infers}[3]{\infer[{[#3]}]{#2}{#1}}

% Contexts
\newcommand{\GammaF}{\Gamma^{+}}  % Focused context
\newcommand{\DeltaF}{\Delta^{+}}  % Focused succedent
\newcommand{\GammaA}{\Gamma^{-}}  % Unfocused (ambient) context
\newcommand{\DeltaA}{\Delta^{-}}  % Unfocused succedent

% Formula polarity
\newcommand{\pos}{^{+}}  % Positive polarity
\newcommand{\neg}{^{-}}  % Negative polarity

% For export to other documents
\ifdefined\THEOREMSTANDALONE
  \usepackage[active,tightpage]{preview}
  \PreviewEnvironment{equation*}
  \PreviewEnvironment{align*}
  \PreviewEnvironment{theorem}
  \PreviewEnvironment{proof}
\fi

% ID tracking for theorem validation
\newcommand{\thmid}[1]{\label{thm:#1}\gdef\CurrentThmID{#1}}
\newcommand{\getthmid}{\CurrentThmID}

% Document info (to be overridden per theorem)
\title{Theorem Document}
\author{Structural Rules Monograph}
\date{\today}

\begin{document}

\ifdefined\THEOREMSTANDALONE
  \maketitle
\fi

% Content begins here


\title{Theorem: Contraction as Resource Identification}
\date{}

\begin{document}

\thmid{contraction-associativity}

\begin{theorem}[Contraction and Associative Resource Composition]
\label{thm:contraction-associativity}
In classical sequent calculus (LK) with full contraction, the contraction rules satisfy an associativity-like property: multiple contractions can be performed in any order with the same result, making the identification of duplicate formulas associative.
\end{theorem}

\begin{proof}
Consider a context with three occurrences of formula $A$: $\Gamma, A, A, A, \Delta$.

We show that contracting the first two, then the result with the third, equals contracting different pairs first.

Using left contraction twice:
\[
\infer[\text{C}_L]{\Gamma, A, \Delta}{\infer[\text{C}_L]{\Gamma, A, A, \Delta}{\Gamma, A, A, A, \Delta}}
\]

The associativity of contraction follows from the structural nature of the rule. Given any multiset of $n$ occurrences of $A$, repeated application of contraction yields exactly one occurrence, regardless of the order of pairwise contractions.

Formally, define $\text{contract}(\Gamma)$ as the result of applying left and right contraction maximally to $\Gamma$. Then for any $\Gamma$ with multiple formula occurrences:
\[
\text{contract}(\text{contract}(\Gamma)) = \text{contract}(\Gamma)
\]
showing that contraction is idempotent, and for different contraction sequences $\sigma_1, \sigma_2$:
\[
\sigma_1(\Gamma) = \sigma_2(\Gamma) = \text{contract}(\Gamma)
\]
demonstrating that contraction order does not affect the final result.
\end{proof}

\begin{corollary}[Contraction as Diagonal]
In categorical semantics, contraction corresponds to the diagonal map $\delta_A : A \to A \times A$. The associativity of contraction reflects the coassociativity of the comonoid structure on $A$ in a cartesian category.
\end{corollary}

\begin{proof}[Proof of Corollary]
In a cartesian closed category, every object $A$ has a canonical comonoid structure with:
\begin{align*}
\delta_A &: A \to A \times A \quad \text{(diagonal/copying)} \\
\epsilon_A &: A \to 1 \quad \text{(weakening/terminal)}
\end{align*}

The coassociativity condition states:
\[
(\delta_A \times \text{id}_A) \circ \delta_A = (\text{id}_A \times \delta_A) \circ \delta_A : A \to A \times A \times A
\]

This categorical property exactly corresponds to the associativity of contraction in logic: copying $A$ twice then copying one result equals copying $A$ then copying the result twice, with both yielding three copies of $A$.
\end{proof}

\begin{remark}[Linear Logic Without Contraction]
In linear logic, where contraction is restricted to exponential modalities $\oc A$:
\begin{itemize}
\item Only formulas marked with $\oc$ can be contracted
\item The diagonal map exists only for $\oc A$, not arbitrary $A$
\item This corresponds to the categorical structure of linear categories
\end{itemize}
\end{remark}

\end{document}
