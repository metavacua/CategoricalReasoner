% Theorem: Admissibility of Contraction in Focused Sequent Calculus
% ID: thm-contraction-admissibility
% Status: Complete

% Minimal preamble for standalone theorem/proof documents
% This preamble is designed for compiling individual theorems

\documentclass[12pt]{article}

% Essential packages
\usepackage{amsmath}
\usepackage{amssymb}
\usepackage{amsthm}

% Theorem environments
\theoremstyle{definition}
\newtheorem{theorem}{Theorem}[section]
\newtheorem{lemma}[theorem]{Lemma}
\newtheorem{corollary}[theorem]{Corollary}
\newtheorem{proposition}[theorem]{Proposition}
\newtheorem{definition}[theorem]{Definition}

\theoremstyle{remark}
\newtheorem{remark}{Remark}

% Proof environment
\usepackage{proof}

% Sequent notation
\newcommand{\sequent}[2]{#1 \vdash #2}
\newcommand{\Sequent}{\vdash}

% Focused sequent calculus notation
\newcommand{\uparrow}{\mathbin{\uparrow}}
\newcommand{\downarrow}{\mathbin{\downarrow}}
\newcommand{\foc}[2]{#1 \mathbin{\Downarrow} #2}  % Focused sequent
\newcommand\focal{
  \mathchoice{\mathbin{\Downarrow}}{\mathbin{\Downarrow}}{\mathbin{\scriptstyle\Downarrow}}{\mathbin{\scriptstyle\Downarrow}}
}
\newcommand{\async}[2]{#1 \uparrow #2}  % Asynchronous phase
\newcommand{\sync}[2]{#1 \downarrow #2}  % Synchronous phase

% Structural rules
\newcommand{\weakening}{\textsf{W}}
\newcommand{\contraction}{\textsf{C}}
\newcommand{\exchange}{\textsf{E}}

% Focused structural rules
\newcommand{\fweakening}{\textsf{W}^{\downarrow}}
\newcommand{\fcontraction}{\textsf{C}^{\downarrow}}
\newcommand{\fexchange}{\textsf{E}^{\downarrow}}

% Inference rules
\newcommand{\infrule}[3]{\dfrac{#1}{#2}\;\;{\scriptsize \textsc{#3}}}
\newcommand{\infers}[3]{\infer[{[#3]}]{#2}{#1}}

% Contexts
\newcommand{\GammaF}{\Gamma^{+}}  % Focused context
\newcommand{\DeltaF}{\Delta^{+}}  % Focused succedent
\newcommand{\GammaA}{\Gamma^{-}}  % Unfocused (ambient) context
\newcommand{\DeltaA}{\Delta^{-}}  % Unfocused succedent

% Formula polarity
\newcommand{\pos}{^{+}}  % Positive polarity
\newcommand{\neg}{^{-}}  % Negative polarity

% For export to other documents
\ifdefined\THEOREMSTANDALONE
  \usepackage[active,tightpage]{preview}
  \PreviewEnvironment{equation*}
  \PreviewEnvironment{align*}
  \PreviewEnvironment{theorem}
  \PreviewEnvironment{proof}
\fi

% ID tracking for theorem validation
\newcommand{\thmid}[1]{\label{thm:#1}\gdef\CurrentThmID{#1}}
\newcommand{\getthmid}{\CurrentThmID}

% Document info (to be overridden per theorem)
\title{Theorem Document}
\author{Structural Rules Monograph}
\date{\today}

\begin{document}

\ifdefined\THEOREMSTANDALONE
  \maketitle
\fi

% Content begins here


\title{Theorem: Admissibility of Contraction}
\date{}

\begin{document}

\thmid{contraction-admissibility}

\begin{theorem}[Admissibility of Contraction in Focused Calculus]
\label{thm:contraction-admissibility}
In the focused sequent calculus with full contraction, the following rules are admissible:

\begin{enumerate}
\item \textbf{Left Contraction (Asynchronous):}
\[
\infer[\fcontractionL]{\Gamma, A \Uparrow \Theta \vdash \Delta}{\Gamma, A, A \Uparrow \Theta \vdash \Delta}
\]

\item \textbf{Right Contraction (Asynchronous):}
\[
\infer[\fcontractionR]{\Gamma \Uparrow \Theta \vdash \Delta, A}{\Gamma \Uparrow \Theta \vdash \Delta, A, A}
\]

\item \textbf{Focused Contraction:}
\[
\infer[\fcontractionF]{\Gamma; \Omega, A \focal C}{\Gamma; \Omega, A, A \focal C}
\]
\end{enumerate}
\end{theorem}

\begin{proof}
By induction on the structure of the derivation, with nested induction on the formula being contracted when necessary.

\textbf{Case 1:} Contraction on atomic formula $P$.
If $P$ is positive and we contract in the focused context:
\[
\infer[\fcontractionF]{\Gamma; P, \Omega \focal C}{\Gamma; P, P, \Omega \focal C}
\]
This is immediate when the first occurrence of $P$ is not principal (simply delete the duplicate). When $P$ is principal, we use the identity expansion property.

\textbf{Case 2:} Contraction on compound formula $A \tensor B$ (positive).
If $A \tensor B$ is in the store during the asynchronous phase:
\[
\infer[\fcontractionL]{\Gamma \Uparrow A \tensor B, \Theta \vdash \Delta}{\Gamma \Uparrow A \tensor B, A \tensor B, \Theta \vdash \Delta}
\]
By applying the asynchronous rule $\tensor L$ to both occurrences, we obtain premises that contract the subformulas, which is admissible by the induction hypothesis.

\textbf{Case 3:} Contraction during synchronous phase.
When contracting in the focused zone:
\[
\infer[\fcontractionF]{\Gamma; A, \Omega \focal C}{\Gamma; A, A, \Omega \focal C}
\]
If the first $A$ is principal, we appeal to the induction hypothesis on the derivation height. If not principal, the contraction is pushed into the context.

\qedhere
\end{proof}

\begin{remark}
The admissibility of contraction in the focused system is more subtle than weakening, as it requires maintaining the focusing structure while eliminating duplicate formulas. The key insight is that contraction preserves the polarity-based focusing discipline.
\end{remark}

\end{document}
