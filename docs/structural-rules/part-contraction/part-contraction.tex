% Part II: Contraction
% ID: part-contraction

\part{Contraction}

\chapter*{Introduction to Contraction}
\label{sec-contraction-intro}
\phantomsection
\addcontentsline{toc}{chapter}{Introduction to Contraction}

Contraction is a structural rule that permits the identification of duplicate formulas within a sequent context. In its classical form, contraction allows the inference from $\Gamma, A, A \vdash C$ to $\Gamma, A \vdash C$, thereby enabling the reuse of premises.

This part examines contraction as the dual of weakening: while weakening creates resources, contraction consumes them through identification. The interplay between weakening and contraction characterizes classical logic, while their restriction leads to substructural logics.

\section*{Additive Contraction}
\label{sec-contraction-additive}
% TODO: \input{sec-contraction-additive-formal}

Additive contraction operates in contexts where duplicate formulas are implicitly merged. In additive sequent calculi, contraction reflects the implicit assumption that premises can be used multiple times without explicit accounting.

\section*{Multiplicative Contraction}
\label{sec-contraction-multiplicative}
% TODO: \input{sec-contraction-multiplicative-formal}

Multiplicative contraction, in contrast, operates in contexts where duplicates must be explicitly managed. In linear logic, contraction is available only for formulas marked with the exponential modality $\oc A$, indicating that the formula may be ``copied'' as needed.

\section*{Contraction-Free Logics}
\label{sec-contraction-free}
% TODO: \input{sec-contraction-free}

Logics that reject contraction include:
\begin{itemize}
    \item \textbf{Linear logic}: Resources cannot be duplicated freely
    \item \textbf{Affine logic}: Weakening admitted, contraction rejected
    \item \textbf{Relevant logic}: Both weakening and contraction rejected
\end{itemize}

\section*{Categorical Semantics}
\label{sec-contraction-categorical}
% TODO: \input{sec-contraction-semantics}

Categorically, contraction corresponds to the existence of \emph{diagonal maps}. In a Cartesian closed category, the natural transformation $\delta_A : A \to A \times A$ witnesses contraction, providing the computational interpretation of ``copying'' or ``duplication.''

% Subsection inputs (to be created as separate files)
% \input{sec-contraction-history}
% \input{sec-contraction-additive-formal}
% \input{sec-contraction-multiplicative-formal}
% \input{sec-contraction-semantics}
% \input{sec-contraction-computational}
