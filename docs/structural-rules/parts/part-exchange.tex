% Part III: Exchange
% ID: part-exchange

\part{Exchange}

\chapter*{Introduction to Exchange}
\label{sec-exchange-intro}
\phantomsection
\addcontentsline{toc}{chapter}{Introduction to Exchange}

Exchange is a structural rule that permits the reordering of formulas within a sequent context. In its classical form, exchange allows the inference from $\Gamma, A, B, \Delta \vdash C$ to $\Gamma, B, A, \Delta \vdash C$, thereby treating contexts as multisets rather than sequences.

This part examines exchange as the most fundamental structural rule: unlike weakening and contraction, exchange does not change the number of formula occurrences, only their arrangement. The rejection of exchange leads to non-commutative logics with rich algebraic structure.

\section*{Permutation of Antecedents}
\label{sec-exchange-antecedent}
% TODO: \input{sec-exchange-antecedent-formal}

Antecedent exchange governs the reordering of premises in the left-hand side of a sequent. In classical and intuitionistic logics, antecedents are treated as unordered collections, making exchange implicit.

\section*{Permutation of Succedents}
\label{sec-exchange-succedent}
% TODO: \input{sec-exchange-succedent-formal}

Succedent exchange governs the reordering of conclusions in the right-hand side of a sequent. In classical logic with multiple conclusions, exchange applies symmetrically to both sides.

\section*{Exchange-Free Logics}
\label{sec-exchange-free}
% TODO: \input{sec-exchange-free}

Logics that reject exchange include:
\begin{itemize}
    \item \textbf{Lambek calculus}: Ordered type logic for natural language
    \item \textbf{Bunched implications}: Separation logic and resource models
    \item \textbf{Non-commutative linear logic}: Ordered resource management
\end{itemize}

\section*{Categorical Semantics}
\label{sec-exchange-categorical}
% TODO: \input{sec-exchange-semantics}

Categorically, exchange corresponds to the existence of \emph{symmetry isomorphisms} or \emph{braidings}. In a symmetric monoidal category, the natural isomorphism $\sigma_{A,B} : A \otimes B \to B \otimes A$ witnesses exchange, while braided monoidal categories permit controlled forms of exchange.

% Subsection inputs (to be created as separate files)
% \input{sec-exchange-history}
% \input{sec-exchange-antecedent-formal}
% \input{sec-exchange-succedent-formal}
% \input{sec-exchange-semantics}
% \input{sec-exchange-computational}
