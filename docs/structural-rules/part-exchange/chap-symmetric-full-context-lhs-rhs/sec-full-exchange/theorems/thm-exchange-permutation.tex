% Theorem: Exchange as Permutation Invariance
% ID: thm-exchange-permutation
% Research Source: DBpedia - Structural proof theory

% Minimal preamble for standalone theorem/proof documents
% This preamble is designed for compiling individual theorems

\documentclass[12pt]{article}

% Essential packages
\usepackage{amsmath}
\usepackage{amssymb}
\usepackage{amsthm}

% Theorem environments
\theoremstyle{definition}
\newtheorem{theorem}{Theorem}[section]
\newtheorem{lemma}[theorem]{Lemma}
\newtheorem{corollary}[theorem]{Corollary}
\newtheorem{proposition}[theorem]{Proposition}
\newtheorem{definition}[theorem]{Definition}

\theoremstyle{remark}
\newtheorem{remark}{Remark}

% Proof environment
\usepackage{proof}

% Sequent notation
\newcommand{\sequent}[2]{#1 \vdash #2}
\newcommand{\Sequent}{\vdash}

% Focused sequent calculus notation
\newcommand{\uparrow}{\mathbin{\uparrow}}
\newcommand{\downarrow}{\mathbin{\downarrow}}
\newcommand{\foc}[2]{#1 \mathbin{\Downarrow} #2}  % Focused sequent
\newcommand\focal{
  \mathchoice{\mathbin{\Downarrow}}{\mathbin{\Downarrow}}{\mathbin{\scriptstyle\Downarrow}}{\mathbin{\scriptstyle\Downarrow}}
}
\newcommand{\async}[2]{#1 \uparrow #2}  % Asynchronous phase
\newcommand{\sync}[2]{#1 \downarrow #2}  % Synchronous phase

% Structural rules
\newcommand{\weakening}{\textsf{W}}
\newcommand{\contraction}{\textsf{C}}
\newcommand{\exchange}{\textsf{E}}

% Focused structural rules
\newcommand{\fweakening}{\textsf{W}^{\downarrow}}
\newcommand{\fcontraction}{\textsf{C}^{\downarrow}}
\newcommand{\fexchange}{\textsf{E}^{\downarrow}}

% Inference rules
\newcommand{\infrule}[3]{\dfrac{#1}{#2}\;\;{\scriptsize \textsc{#3}}}
\newcommand{\infers}[3]{\infer[{[#3]}]{#2}{#1}}

% Contexts
\newcommand{\GammaF}{\Gamma^{+}}  % Focused context
\newcommand{\DeltaF}{\Delta^{+}}  % Focused succedent
\newcommand{\GammaA}{\Gamma^{-}}  % Unfocused (ambient) context
\newcommand{\DeltaA}{\Delta^{-}}  % Unfocused succedent

% Formula polarity
\newcommand{\pos}{^{+}}  % Positive polarity
\newcommand{\neg}{^{-}}  % Negative polarity

% For export to other documents
\ifdefined\THEOREMSTANDALONE
  \usepackage[active,tightpage]{preview}
  \PreviewEnvironment{equation*}
  \PreviewEnvironment{align*}
  \PreviewEnvironment{theorem}
  \PreviewEnvironment{proof}
\fi

% ID tracking for theorem validation
\newcommand{\thmid}[1]{\label{thm:#1}\gdef\CurrentThmID{#1}}
\newcommand{\getthmid}{\CurrentThmID}

% Document info (to be overridden per theorem)
\title{Theorem Document}
\author{Structural Rules Monograph}
\date{\today}

\begin{document}

\ifdefined\THEOREMSTANDALONE
  \maketitle
\fi

% Content begins here


\title{Theorem: Exchange as Permutation Invariance}
\date{}

\begin{document}

\thmid{exchange-permutation}

\begin{theorem}[Structural Exchange as Permutation Invariance]
\label{thm:exchange-permutation}
In classical sequent calculus (LK), the exchange rule provides complete permutation invariance: if $\Gamma \vdash \Delta$ is derivable, then $\Gamma' \vdash \Delta'$ is derivable for any permutations $\Gamma'$ of $\Gamma$ and $\Delta'$ of $\Delta$.
\end{theorem}

\begin{proof}
We prove by induction on the length of the derivation.

\textbf{Base Case:} Axiom rule $\overline{A \vdash A}$
Any permutation of $A \vdash A$ is identical to itself, so the property holds trivially.

\textbf{Inductive Step:} Consider a rule application in the derivation.

For left rules, suppose the last rule is:
\[
\infer[\land L_i]{\Gamma, A \land B, \Delta \vdash \Sigma}{\Gamma, A, \Delta \vdash \Sigma}
\]

By the induction hypothesis, $\Gamma, A, \Delta \vdash \Sigma$ is invariant under permutation of its antecedent. The exchange rules allow us to permute $A \land B$ with any formula in $\Gamma$ or $\Delta$, establishing that the conclusion is also permutation invariant.

For right rules, the argument is symmetric using right exchange.

The key insight is that the exchange rules:
\[
\infer[\text{Exchange}_L]{\Gamma, B, A, \Delta \vdash \Sigma}{\Gamma, A, B, \Delta \vdash \Sigma}
\quad
\infer[\text{Exchange}_R]{\Gamma \vdash \Delta, B, A, \Sigma}{\Gamma \vdash \Delta, A, B, \Sigma}
\]
generate the full symmetric group on formula positions. Since any permutation can be decomposed into adjacent transpositions, and exchange provides exactly these transpositions, complete permutation invariance follows.
\end{proof}

\begin{corollary}[Multiset Semantics]
The exchange rule makes sequent contexts behave as multisets (where order does not matter) rather than sequences or lists.
\end{corollary}

\begin{remark}[Non-Commutative Variants]
When exchange is rejected:
\begin{itemize}
\item \textbf{Lambek Calculus:} Sequents are ordered; formula position matters semantically.
\item \textbf{Bunched Implications:} Combines commutative and non-commutative contexts using tree structures.
\item \textbf{Categorical Model:} Without exchange, categories are monoidal but not necessarily symmetric.
\end{itemize}
\end{remark}

\begin{remark}[Duality]
The duality between left and right exchange mirrors the symmetry of the sequent arrow. In category theory, this corresponds to the symmetry isomorphism $\sigma_{A,B} : A \otimes B \cong B \otimes A$ in symmetric monoidal categories.
\end{remark}

\end{document}
