% Theorem: Deduction Theorem in Classical Logic (LK)
% ID: thm-deduction-lk
% Variant: Full (Classical LK)

% Minimal preamble for standalone theorem/proof documents
% This preamble is designed for compiling individual theorems

\documentclass[12pt]{article}

% Essential packages
\usepackage{amsmath}
\usepackage{amssymb}
\usepackage{amsthm}

% Theorem environments
\theoremstyle{definition}
\newtheorem{theorem}{Theorem}[section]
\newtheorem{lemma}[theorem]{Lemma}
\newtheorem{corollary}[theorem]{Corollary}
\newtheorem{proposition}[theorem]{Proposition}
\newtheorem{definition}[theorem]{Definition}

\theoremstyle{remark}
\newtheorem{remark}{Remark}

% Proof environment
\usepackage{proof}

% Sequent notation
\newcommand{\sequent}[2]{#1 \vdash #2}
\newcommand{\Sequent}{\vdash}

% Focused sequent calculus notation
\newcommand{\uparrow}{\mathbin{\uparrow}}
\newcommand{\downarrow}{\mathbin{\downarrow}}
\newcommand{\foc}[2]{#1 \mathbin{\Downarrow} #2}  % Focused sequent
\newcommand\focal{
  \mathchoice{\mathbin{\Downarrow}}{\mathbin{\Downarrow}}{\mathbin{\scriptstyle\Downarrow}}{\mathbin{\scriptstyle\Downarrow}}
}
\newcommand{\async}[2]{#1 \uparrow #2}  % Asynchronous phase
\newcommand{\sync}[2]{#1 \downarrow #2}  % Synchronous phase

% Structural rules
\newcommand{\weakening}{\textsf{W}}
\newcommand{\contraction}{\textsf{C}}
\newcommand{\exchange}{\textsf{E}}

% Focused structural rules
\newcommand{\fweakening}{\textsf{W}^{\downarrow}}
\newcommand{\fcontraction}{\textsf{C}^{\downarrow}}
\newcommand{\fexchange}{\textsf{E}^{\downarrow}}

% Inference rules
\newcommand{\infrule}[3]{\dfrac{#1}{#2}\;\;{\scriptsize \textsc{#3}}}
\newcommand{\infers}[3]{\infer[{[#3]}]{#2}{#1}}

% Contexts
\newcommand{\GammaF}{\Gamma^{+}}  % Focused context
\newcommand{\DeltaF}{\Delta^{+}}  % Focused succedent
\newcommand{\GammaA}{\Gamma^{-}}  % Unfocused (ambient) context
\newcommand{\DeltaA}{\Delta^{-}}  % Unfocused succedent

% Formula polarity
\newcommand{\pos}{^{+}}  % Positive polarity
\newcommand{\neg}{^{-}}  % Negative polarity

% For export to other documents
\ifdefined\THEOREMSTANDALONE
  \usepackage[active,tightpage]{preview}
  \PreviewEnvironment{equation*}
  \PreviewEnvironment{align*}
  \PreviewEnvironment{theorem}
  \PreviewEnvironment{proof}
\fi

% ID tracking for theorem validation
\newcommand{\thmid}[1]{\label{thm:#1}\gdef\CurrentThmID{#1}}
\newcommand{\getthmid}{\CurrentThmID}

% Document info (to be overridden per theorem)
\title{Theorem Document}
\author{Structural Rules Monograph}
\date{\today}

\begin{document}

\ifdefined\THEOREMSTANDALONE
  \maketitle
\fi

% Content begins here


\title{Theorem: Deduction Theorem in LK}
\date{}

\begin{document}

\thmid{deduction-lk}

\begin{theorem}[Deduction Theorem in Classical Sequent Calculus LK]
\label{thm:deduction-lk}
For any formulas $A$ and $B$ in classical propositional logic LK with full weakening and contraction:
\[
\Gamma, A \vdash B \quad \text{if and only if} \quad \Gamma \vdash A \rightarrow B
\]
where weakening and contraction are freely available on both sides.
\end{theorem}

\begin{proof}
We prove both directions using root-first decomposition.

$(\Rightarrow)$ Assume $\Gamma, A \vdash_{LK} B$. We construct $\Gamma \vdash_{LK} A \rightarrow B$.

\[
\infer[\rightarrow_R]{\Gamma \vdash A \rightarrow B}{
    \infer[{[W_R]}]{\Gamma, A \vdash B}{\Gamma, A \vdash B}}
\]

The derivation uses implication right rule, with weakening on the right to manage contexts.

$(\Leftarrow)$ Assume $\Gamma \vdash_{LK} A \rightarrow B$. We construct $\Gamma, A \vdash_{LK} B$.

\[
\infer[\text{Cut}]{\Gamma, A \vdash B}{
    \infer[\text{Id}]{A \vdash A}{} & 
    \infer[{[W_L]}]{\Gamma, A \rightarrow B, A \vdash B}{
        \infer[\rightarrow_L]{\Gamma, A \rightarrow B \vdash B}{
            \Gamma \vdash A, B & 
            \infer[\text{Id}]{B \vdash B}{}}}}
\]

The proof uses cut with identity axioms and structural rules available in LK.
\end{proof}

\begin{remark}
This theorem holds in LK with full structural rules. For substructural variants (Linear, Affine, Relevant), see the corresponding sections under Linear Weakening, Affine Weakening, and Relevant Weakening.
\end{remark}

\end{document}
