\chapter*{Abstract}
\addcontentsline{toc}{chapter}{Abstract}

This white paper presents the logic-witness formalism, a rigorous metamathematical framework for modeling formal logics as objects in a two-dimensional lattice category. The foundation rests on sequent calculi presentations and physical realizability constraints derived from quantum information theory.

\textbf{Core Contribution:} We establish a categorical structure where logics are organized along two orthogonal dimensions:

\begin{enumerate}
\item \textbf{Horizontal dimension}: Antecedent/succedent restrictions within logical families (classical, intuitionistic, dual-intuitionistic, monotonic)
\item \textbf{Vertical dimension}: Structural rule presence/absence across families (exchange, weakening, contraction)
\end{enumerate}

This two-dimensional structure forms a lattice of lattices, with morphisms representing extension and interpretability relationships. Each logic is characterized by a \emph{logical signature} (structural rules, sequent restrictions, connectives) and a \emph{theoretical signature} (meta-predicates, axioms, theorems).

\textbf{Physical Grounding:} The framework is grounded in physical realizability. Quantum-safe logics (those without weakening or contraction) embody the no-cloning and no-erasure theorems of quantum mechanics. Non-structural logics represent maximally resource-constrained computation.

\textbf{Witness Formalism:} Through the Curry-Howard-Kleene-Lambek correspondence, each logic-theory pairing instantiates as an executable program or physical circuit. The \emph{witness functor} maps formal specifications to constructive implementations, making the formalism directly realizable in software and hardware.

\textbf{Scope:} This white paper focuses on establishing the correct abstract categorical structure. Formal proofs, complete commutative diagrams, and mechanized verification are identified as vacancies for future work. Robinson arithmetic serves as the canonical case study across all logic variants.

\textbf{Applications:} The formalism enables systematic theory transformation under logic changes, provides a foundation for quantum-safe programming languages, and offers a unified framework for comparing proof-theoretic strength across logical systems.
