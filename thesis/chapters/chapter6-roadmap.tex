\chapter{Technical Roadmap and Future Work}

\section{Development Phases}

\subsection{Phase 1: Formalize Robinson in RDF for All Canonical Logics}

\textbf{Objective}: Create RDF specifications for Robinson arithmetic in all canonical logics.

\textbf{Tasks}:
\begin{enumerate}
\item Specify Classical Robinson (LK + Robinson) in TTL
\item Specify Intuitionistic Robinson (LJ + Robinson) in TTL
\item Specify Dual-Intuitionistic Robinson (LDJ + Robinson) in TTL
\item Specify Monotonic Robinson (Monotonic + restricted Robinson) in TTL
\item Specify Linear Robinson (LL + resource-aware Robinson) in TTL
\end{enumerate}

\textbf{Deliverable}: Five RDF files in \texttt{ontologies/examples/robinson/}

\textbf{Validation}: All RDF files parse without errors; SHACL validation passes.

\subsection{Phase 2: Implement RDF Parser and Code Generator Prototype}

\textbf{Objective}: Build a prototype code generator that produces executable programs from RDF specifications.

\textbf{Tasks}:
\begin{enumerate}
\item Implement RDF parser (Python + RDFLib)
\item Extract logic-theory pairing from RDF graph
\item Implement witness functor for Classical Robinson (target: Python or Haskell)
\item Generate executable program and verify correctness (manual testing)
\end{enumerate}

\textbf{Deliverable}: Python script \texttt{tools/generate-witness.py} that generates code from RDF.

\textbf{Validation}: Classical Robinson witness executes correctly; $2 + 3 = 5$ computes as expected.

\subsection{Phase 3: Extend Code Generator to All Logic Variants}

\textbf{Objective}: Generalize the code generator to all canonical logics.

\textbf{Tasks}:
\begin{enumerate}
\item Implement witness functor for Intuitionistic Robinson (constructive code generation)
\item Implement witness functor for Linear Robinson (resource-aware code generation)
\item Implement witness functor for Monotonic Robinson (positive operations only)
\item Extend to Dual-Intuitionistic Robinson (co-constructive code generation)
\end{enumerate}

\textbf{Deliverable}: Code generator supports all five Robinson variants.

\textbf{Validation}: All five witnesses execute correctly; behavioral differences are documented.

\subsection{Phase 4: Proof of Execution (Behavioral Equivalence and Differences)}

\textbf{Objective}: Demonstrate algorithmic equivalence and behavioral differences across logic variants.

\textbf{Tasks}:
\begin{enumerate}
\item Define test suite (arithmetic operations: addition, multiplication, comparisons)
\item Execute test suite on all five Robinson witnesses
\item Document algorithmic equivalence (same results)
\item Document behavioral differences (proof structures, resource usage, decidability)
\end{enumerate}

\textbf{Deliverable}: Technical report comparing all five Robinson variants.

\textbf{Validation}: Test suite passes for all variants; behavioral differences are formalized.

\subsection{Phase 5: Hardware Synthesis (Quantum-Safe Logics)}

\textbf{Objective}: Synthesize hardware specifications (quantum circuits) from quantum-safe logic specifications.

\textbf{Tasks}:
\begin{enumerate}
\item Extend code generator to target quantum assembly (e.g., OpenQASM)
\item Generate quantum circuits from Linear Robinson specification
\item Validate no-cloning and no-erasure constraints in generated circuits
\item Test on quantum simulator (Qiskit, Cirq)
\end{enumerate}

\textbf{Deliverable}: Quantum circuit generator for linear logic specifications.

\textbf{Validation}: Generated circuits respect quantum constraints; execute correctly on simulators.

\subsection{Phase 6: Generalization Beyond Robinson}

\textbf{Objective}: Extend the formalism to theories beyond Robinson arithmetic.

\textbf{Tasks}:
\begin{enumerate}
\item Specify Peano Arithmetic (PA) and Heyting Arithmetic (HA) in RDF
\item Specify Euclidean Geometry in RDF (for all canonical logics)
\item Generalize code generator to arbitrary theories
\item Document theory transformation procedures formally
\end{enumerate}

\textbf{Deliverable}: General-purpose synthesis engine for arbitrary logic-theory pairings.

\textbf{Validation}: PA and HA witnesses execute correctly; geometry proofs compile to programs.

\section{Implementation Strategy Analysis}

\subsection{Python + RDFLib}

\textbf{Strengths}:
\begin{itemize}
\item Rapid prototyping
\item Extensive RDF ecosystem (RDFLib, pySHACL)
\item Easy integration with scientific computing libraries (NumPy, SciPy)
\item Suitable for research and experimentation
\end{itemize}

\textbf{Weaknesses}:
\begin{itemize}
\item Performance limitations for large-scale synthesis
\item Less suitable for production deployment
\end{itemize}

\textbf{Use Case}: Prototype witness functor and validate approach.

\subsection{Java + Apache Jena}

\textbf{Strengths}:
\begin{itemize}
\item Production-grade RDF processing
\item Robust SPARQL engine and reasoning capabilities
\item Scalable to large knowledge graphs
\item Suitable for industrial deployment
\end{itemize}

\textbf{Weaknesses}:
\begin{itemize}
\item Slower iteration cycle (compilation overhead)
\item More verbose than Python
\end{itemize}

\textbf{Use Case}: Production synthesis engine for large-scale code generation.

\subsection{Hybrid Approach}

\textbf{Recommendation}: Prototype in Python, deploy in Java.

\begin{itemize}
\item \textbf{Phase 1--3}: Rapid prototyping with Python + RDFLib
\item \textbf{Phase 4--6}: Migrate to Java + Apache Jena for production
\item \textbf{Rationale}: Fast iteration during research phase; robust deployment for production
\end{itemize}

\section{Mechanized Verification Integration}

\subsection{Isabelle/AFP}

\textbf{Objective}: Formalize the categorical structure and key theorems in Isabelle.

\textbf{Tasks}:
\begin{enumerate}
\item Define category of categorizable logics in Isabelle/HOL
\item Formalize two-dimensional lattice structure
\item Prove meta-theorems (fragment containment, diamond commutativity)
\item Integrate with Archive of Formal Proofs (AFP)
\end{enumerate}

\textbf{Deliverable}: Isabelle formalization of Chapters 2--3.

\textbf{Validation}: All theorems mechanically verified; AFP submission accepted.

\subsection{Coq}

\textbf{Objective}: Constructive formalization of intuitionistic and linear logics.

\textbf{Tasks}:
\begin{enumerate}
\item Define intuitionistic Robinson in Coq
\item Define linear Robinson in Coq
\item Extract executable code from Coq proofs (witness programs)
\item Compare extracted code with generated witnesses from RDF
\end{enumerate}

\textbf{Deliverable}: Coq formalization of intuitionistic and linear Robinson.

\textbf{Validation}: Extracted programs are behaviorally equivalent to RDF-generated witnesses.

\subsection{Lean}

\textbf{Objective}: Leverage Lean's MathLib for categorical formalizations.

\textbf{Tasks}:
\begin{enumerate}
\item Define logics as objects in category theory framework (MathLib)
\item Formalize morphisms and commutative diagrams
\item Integrate with Lean 4's proof automation
\end{enumerate}

\textbf{Deliverable}: Lean formalization of categorical structure.

\textbf{Validation}: All categorical axioms mechanically verified.

\section{Long-Term Vision}

\subsection{Quantum-Safe Programming Languages}

\textbf{Vision}: A programming language family based on the logic-witness formalism.

\textbf{Characteristics}:
\begin{itemize}
\item \textbf{Linear Type System}: Enforces no-cloning and no-erasure
\item \textbf{Quantum-Safe by Construction}: Programs are guaranteed to respect quantum constraints
\item \textbf{Multiple Logic Backends}: Switch between classical, intuitionistic, linear, and monotonic semantics
\item \textbf{Verified Compilation}: Generated code is correct by construction (witness functor)
\end{itemize}

\textbf{Example Languages}:
\begin{itemize}
\item \textbf{Catty-Haskell}: Linear Haskell extension with quantum-safe types
\item \textbf{Catty-Rust}: Rust extension with formal logic backend
\item \textbf{Catty-Q}: Quantum programming language with linear logic semantics
\end{itemize}

\subsection{Hardware Synthesis for Quantum Computing}

\textbf{Vision}: A synthesis toolchain for generating quantum circuits from formal specifications.

\textbf{Workflow}:
\[ \text{Logic-Theory Spec (RDF)} \to \text{Witness Functor} \to \text{Quantum Circuit (QASM)} \to \text{Quantum Hardware} \]

\textbf{Benefits}:
\begin{itemize}
\item No-cloning and no-erasure enforced by linear types
\item Verified correctness (categorical semantics)
\item Automatic optimization (based on categorical properties)
\end{itemize}

\subsection{Proof-Carrying Code}

\textbf{Vision}: Executable programs bundled with machine-checkable proofs of correctness.

\textbf{Approach}:
\begin{itemize}
\item Witness programs include embedded proof certificates
\item Proof certificates are RDF graphs (provenance, logical axioms, categorical properties)
\item Consumers validate proof certificates before execution
\end{itemize}

\textbf{Application}: Critical systems (aerospace, finance, healthcare) requiring verified correctness.

\section{Community Contributions}

\subsection{Issue-Driven Development}

\textbf{Workflow}:
\begin{enumerate}
\item Identify vacancies (missing proofs, theorems, implementations)
\item Create GitHub issue with clear specification
\item Contributor submits pull request addressing issue
\item Automated validation (LaTeX builds, RDF parses, tests pass)
\item Merge upon approval
\end{enumerate}

\subsection{Contribution Pathways}

\textbf{1. Theorem Formalization}:
\begin{itemize}
\item Prove meta-theorems identified as vacancies
\item Mechanize proofs in Isabelle, Coq, or Lean
\item Document proofs in LaTeX chapters
\end{itemize}

\textbf{2. Ontology Extension}:
\begin{itemize}
\item Add new logics to the knowledge graph
\item Specify morphisms between logic families
\item Validate SHACL constraints
\end{itemize}

\textbf{3. Code Generation}:
\begin{itemize}
\item Extend witness functor to new target languages
\item Implement optimizations for code generation
\item Add hardware synthesis backends
\end{itemize}

\textbf{4. Documentation}:
\begin{itemize}
\item Expand chapters with examples and proofs
\item Create tutorials for using the synthesis engine
\item Write blog posts or technical reports
\end{itemize}

\section{Success Metrics}

\subsection{Phase 1--3 (Foundation)}

\begin{itemize}
\item Robinson arithmetic formalized in all five canonical logics
\item Code generator produces executable witnesses for all five variants
\item Behavioral differences documented and validated
\end{itemize}

\subsection{Phase 4--6 (Generalization)}

\begin{itemize}
\item Witness functor generalizes to arbitrary theories
\item Quantum circuit synthesis works for linear logic specifications
\item At least three non-arithmetic theories formalized (geometry, set theory, type theory)
\end{itemize}

\subsection{Long-Term (Mechanization and Deployment)}

\begin{itemize}
\item Isabelle formalization of categorical structure accepted to AFP
\item At least one quantum-safe programming language prototype operational
\item Production synthesis engine deployed (Java + Apache Jena)
\item Community contributions: 10+ merged pull requests from external contributors
\end{itemize}
