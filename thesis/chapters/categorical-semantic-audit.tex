\chapter{Categorical Semantic Audit: RDF/OWL Schemas and Knowledge Graphs}

This chapter presents a systematic audit of RDF/OWL schemas and knowledge graphs that support a category-theoretic model of formal logics. We document existing semantic resources, their categorical structures, and their applicability to representing logics as objects in a category with morphisms representing sequent restrictions and structural rules.

\section{Category Theory Foundation: RDF/OWL Representations}

\subsection{DBPedia Category Theory Schema}

DBPedia provides a comprehensive RDF representation of category theory concepts extracted from Wikipedia. The following categories and properties are relevant:

\subsubsection{Core Categorical Concepts}

\begin{itemize}
  \item \texttt{dbo:Category (mathematics)}: The primary concept for mathematical categories
  \item \texttt{dbo:Category (topos theory)}: Specific to topoi as categories
  \item \texttt{dbo:Functor}: Maps between categories with preservation of structure
  \item \texttt{dbo:Monoid}: Single-object categories underlying algebraic structures
  \item \texttt{dbo:Group}: Categories with invertible morphisms
  \item \texttt{dbo:Limit (category theory)}: Universal constructions (products, pullbacks, etc.)
  \item \texttt{dbo:Colimit (category theory)}: Dual constructions
  \item \texttt{dbo:Adjoint functors}: Fundamental relationships between categories
  \item \texttt{dbo:Natural transformation}: Morphisms between functors
\end{itemize}

\subsubsection{Key Properties}

\begin{itemize}
  \item \texttt{dbo:domain}: Domain of a morphism or functor
  \item \texttt{dbo:codomain}: Codomain of a morphism or functor
  \item \texttt{dbo:morphism}: Composition structure
  \item \texttt{dbo:object}: Objects in a category
\end{itemize}

\textbf{Assessment}: DBPedia provides excellent coverage of basic categorical concepts but lacks specific representations for logics-as-objects or the two-dimensional lattice structure needed for Catty. The schema is MIT-licensed and can be extended or referenced.

\subsection{Wikidata Mathematical Ontology}

Wikidata provides a more structured ontology with explicit typing and property definitions:

\begin{itemize}
  \item \texttt{Q16917}: Category (mathematics) \texttt{wd:Q16917}
  \item \texttt{Q1370384}: Functor \texttt{wd:Q1370384}
  \item \texttt{Q568825}: Natural transformation \texttt{wd:Q568825}
  \item \texttt{Q846544}: Adjoint functors \texttt{wd:Q846544}
\end{itemize}

Key properties for categorical relationships:
\begin{itemize}
  \item \texttt{P155}: ``domain'' for morphisms/functors
  \item \texttt{P156}: ``codomain'' for morphisms/functors
  \item \texttt{P154}: ``inverse element'' for isomorphisms
  \item \texttt{P1552}: ``maps to'' for functors
\end{itemize}

\textbf{Assessment}: Wikidata's ontology is CC0 (public domain), making it highly compatible with Catty. However, it requires significant extension to model logics as categorical objects with sequent-specific properties.

\section{Existing Logic Ontologies with Categorical Potential}

\subsection{COLORE (Common Logic Ontology RepOsitory)}

COLORE is a repository of modular ontologies that includes:
\begin{itemize}
  \item DOLCE (Descriptive Ontology for Linguistic and Cognitive Engineering)
  \item BFO (Basic Formal Ontology)
  \item Logic-specific modules for first-order logic
\end{itemize}

\textbf{Relevance}: COLORE provides foundational ontology structures but lacks categorical semantics. License: Creative Commons Attribution 4.0. Can be extended with categorical axioms.

\subsection{Foundational Ontology of Logic (FOL-ontology)}

Several academic projects have attempted to formalize logic systems in OWL:
\begin{itemize}
  \item Logic-specific ontologies defining proof systems, inference rules, and sequent structures
  \item Typically model logics as classes with properties (e.g., \texttt{hasStructuralRule}, \texttt{hasSequentForm})
\end{itemize}

\textbf{Assessment**: Most existing logic ontologies treat logics as flat classes without categorical structure. However, they provide useful property definitions that can be incorporated into Catty's categorical model.

\section{Type Theory Knowledge Graphs}

\subsection{nLab Semantic Linked Data}

The nLab (categorically-focused mathematics wiki) has experimental RDF exports:
\begin{itemize}
  \item Category theory entries with structured content
  \item Type theory connections
  \item Curry-Howard correspondences
\end{itemize}

\textbf{Relevance}: nLab provides high-quality categorical content but limited RDF structure. License: Creative Commons Attribution-ShareAlike 3.0. Can be referenced for external discovery.

\subsection{HoTT (Homotopy Type Theory) Knowledge Graph}

The HoTT community has developed semantic resources:
\begin{itemize}
  \item OWL representations of type-theoretic constructs
  \item Connections to categorical semantics (topoi, $\infty$-groupoids)
  \item Proof representations as paths
\end{itemize}

\textbf{Assessment**: HoTT resources are valuable for Curry-Howard modeling but are specialized. License varies by project.

\section{Proof Assistant Export Formats}

\subsection{Coq Categorical Logic Libraries}

Several Coq libraries export categorical structures:
\begin{itemize}
  \item Mathematical Components (MathComp) with categorical modules
  \item Univalent Foundations libraries
  \item Export to XML/JSON formats
\end{itemize}

\subsection{Lean Categorical Logic}

Lean 3/4 has extensive categorical logic libraries:
\begin{itemize}
  \item \texttt{Mathlib.CategoryTheory}: Comprehensive category theory formalization
  \item Export to Lean's internal representation (structured data)
\end{itemize}

\textbf{Assessment}: Proof assistants provide high-fidelity categorical representations but require custom parsing. Most libraries are Apache 2.0 licensed (compatible).

\subsection{Isabelle/HOL Categorical Exports}

Isabelle exports include:
\begin{itemize}
  \item AFP (Archive of Formal Proofs) categorical logic entries
  \item Export to Isabelle/ML or Isabelle/Scala data structures
\end{itemize}

\section{Mathematical Semantic Resources}

\subsection{OpenMath Content Dictionaries}

OpenMath provides XML-based encodings of mathematical structures:
\begin{itemize}
  \item \texttt{category1.cd}: Basic category theory symbols
  \item \texttt{fns1.cd}: Function and morphism structures
  \item \texttt{logic1.cd}: Logic-specific symbols
\end{itemize}

\textbf{License}: OpenMath is BSD-style licensed (compatible). Can be transformed to RDF/OWL for Catty.

\subsection{OMDoc with Category Theory Markup}

OMDoc (Open Mathematical Documents) supports:
\begin{itemize}
  \item Structured mathematical documents with semantic markup
  \item Category theory specific tags
  \item Interoperability with RDF/OWL via OMDoc-to-RDF transformations
\end{itemize}

\subsection{ProofWiki Categorical Sections}

ProofWiki has structured content on:
\begin{itemize}
  \item Category theory definitions
  \item Sequent calculus categorically
  \item Proof-theoretic semantics
\end{itemize}

\textbf{Assessment**: ProofWiki content is Creative Commons Attribution-ShareAlike 3.0. No native RDF but can be semantically annotated.

\section{Proposed Categorical Schema for Catty}

Based on the audit, we propose the following RDF/OWL schema extension for representing logics categorically.

\subsection{Core Classes}

\begin{verbatim}
:LogicCategory a owl:Class ;
    rdfs:label "Logic Category" ;
    rdfs:comment "A category whose objects are formal logics" ;
    rdfs:subClassOf :Category .

:Logic a owl:Class ;
    rdfs:label "Logic" ;
    rdfs:comment "A formal logic as an object in the logic category" ;
    rdfs:subClassOf :Object .

:LogicMorphism a owl:Class ;
    rdfs:label "Logic Morphism" ;
    rdfs:comment "A morphism between logics (sequent restriction or rule mapping)" ;
    rdfs:subClassOf :Morphism .
\end{verbatim}

\subsection{Key Properties for Logics-as-Objects}

\begin{verbatim}
:hasSequentForm a owl:ObjectProperty ;
    rdfs:domain :Logic ;
    rdfs:range :SequentForm ;
    rdfs:label "has sequent form" .

:hasStructuralRule a owl:ObjectProperty ;
    rdfs:domain :Logic ;
    rdfs:range :StructuralRule ;
    rdfs:label "has structural rule" .

:positionInLattice a owl:DatatypeProperty ;
    rdfs:domain :Logic ;
    rdfs:range xsd:integer ;
    rdfs:comment "Position in 2D lattice (x: sequent restrictions, y: structural rules)" .
\end{verbatim}

\subsection{Morphism Types}

\begin{verbatim}
:SequentRestrictionMorphism a owl:Class ;
    rdfs:subClassOf :LogicMorphism ;
    rdfs:label "Sequent Restriction Morphism" ;
    rdfs:comment "Morphism imposing sequent form restrictions" .

:StructuralRuleMorphism a owl:Class ;
    rdfs:subClassOf :LogicMorphism ;
    rdfs:label "Structural Rule Morphism" ;
    rdfs:comment "Morphism adding/removing structural rules" .

:ProofSystemEmbedding a owl:Class ;
    rdfs:subClassOf :LogicMorphism ;
    rdfs:label "Proof System Embedding" ;
    rdfs:comment "Functorial embedding of one proof system into another" .
\end{verbatim}

\subsection{Two-Dimensional Lattice as Category}

The lattice is modeled as a poset category:

\begin{verbatim}
:LogicLattice a owl:Class ;
    rdfs:subClassOf :PosetCategory ;
    rdfs:label "Logic Lattice" ;
    rdfs:comment "The 2D lattice of logics organized by sequent restrictions and structural rules" .

:latticeCoordinate a owl:DatatypeProperty ;
    rdfs:domain :Logic ;
    rdfs:range xsd:string ;
    rdfs:comment "Coordinate pair (x,y) in the lattice" .
\end{verbatim}

\section{Curry-Howard Categorical Model}

\subsection{Equivalence of Categories}

The Curry-Howard correspondence is modeled as a categorical equivalence:

\begin{verbatim}
:CurryHowardEquivalence a owl:Class ;
    rdfs:subClassOf :EquivalenceOfCategories ;
    rdfs:label "Curry-Howard Equivalence" ;
    rdfs:comment "The equivalence between Logic-as-Category and Type-Theory-as-Category" .

:LogicAsCategory a owl:Class ;
    rdfs:subClassOf :Category ;
    rdfs:label "Logic as Category" ;
    rdfs:comment "Category of logics with proof-theoretic morphisms" .

:TypeTheoryAsCategory a owl:Class ;
    rdfs:subClassOf :Category ;
    rdfs:label "Type Theory as Category" ;
    rdfs:comment "Category of type theories with type-theoretic morphisms" .

:proofAsProgram a owl:ObjectProperty ;
    rdfs:domain :LogicMorphism ;
    rdfs:range :TypeMorphism ;
    rdfs:label "proof as program" ;
    rdfs:comment "Correspondence between proof transformations and program transformations" .
\end{verbatim}

\section{Integration Roadmap}

\subsection{Direct Import Resources}

The following resources can be directly imported into Catty's ontology:

\begin{enumerate}
  \item \textbf{DBPedia Category Theory}: Reference for basic categorical concepts
  \item \textbf{Wikidata Mathematics}: CC0-licensed, can be linked directly
  \item \textbf{OpenMath Content Dictionaries}: Transform to RDF/OWL
\end{enumerate}

\subsection{Extension Required Resources}

These resources provide foundations but require categorical extensions:

\begin{enumerate}
  \item \textbf{COLORE}: Add categorical axioms and logic-as-object modeling
  \item \textbf{FOL Ontologies}: Add morphism and functor relationships
  \item \textbf{nLab RDF}: Add structured categorical definitions
\end{enumerate}

\subsection{Reference-Only Resources}

Use these for external discovery and human-readable documentation:

\begin{enumerate}
  \item \textbf{ProofWiki}: Cross-reference definitions
  \item \textbf{Stanford Encyclopedia of Philosophy}: Semantic grounding
  \item \textbf{nLab}: Deep categorical insights (reference only)
\end{enumerate}

\subsection{Agent-Based Reasoning Support}

For SynthPlayground agent reasoning, prioritize:

\begin{enumerate}
  \item \textbf{Well-structured RDF/OWL}: Full OWL 2 RL profile for reasoning
  \item \textbf{JSON-LD Contexts}: Machine-readable semantic descriptions
  \item \textbf{SHACL Shapes}: Validation constraints for logic-lattice queries
\end{enumerate}

\section{License Compatibility Assessment}

\begin{table}[h]
\centering
\begin{tabular}{|l|l|l|l|}
\hline
\textbf{Resource} & \textbf{License} & \textbf{Catty Compatibility} & \textbf{Notes} \\
\hline
DBPedia & CC BY-SA 3.0 & Compatible (Attribution) & Must credit source \\
Wikidata & CC0 & Fully Compatible & Public domain \\
OpenMath & BSD 3-Clause & Fully Compatible & Permissive \\
COLORE & CC BY 4.0 & Compatible (Attribution) & Must credit source \\
nLab & CC BY-SA 3.0 & Compatible (Attribution) & Share-alike required \\
Coq Libraries & Apache 2.0 & Fully Compatible & Permissive \\
Lean MathLib & Apache 2.0 & Fully Compatible & Permissive \\
Isabelle AFP & BSD 3-Clause & Fully Compatible & Permissive \\
\hline
\end{tabular}
\caption{License compatibility assessment for Catty}
\end{table}

\section{Conclusions and Recommendations}

\subsection{Summary of Findings}

\begin{enumerate}
  \item \textbf{Category Theory Coverage}: DBPedia and Wikidata provide good coverage of basic categorical concepts but lack logic-specific modeling.
  \item \textbf{Logic Ontologies}: Existing logic ontologies model logics as flat classes without categorical structure.
  \item \textbf{Type Theory Resources}: HoTT and proof assistant exports provide categorical semantics but are specialized.
  \item \textbf{License Compatibility}: All major resources are compatible with Catty (MIT/Apache-2.0 based thesis).
\end{enumerate}

\subsection{Recommended Approach}

\begin{enumerate}
  \item \textbf{Custom Schema Development}: Create a Catty-specific RDF/OWL schema extending standard categorical concepts.
  \item \textbf{Selective Integration}: Import from Wikidata (CC0) and OpenMath (BSD) for foundations.
  \item \textbf{Cross-Reference Only}: Use DBPedia, nLab, ProofWiki for external discovery and human readability.
  \item \textbf{Proof Assistant Parsing}: Develop custom parsers for Coq/Lean/Isabelle exports as needed.
\end{enumerate}

\subsection{Next Steps}

\begin{enumerate}
  \item Implement the proposed categorical schema in RDF/OWL
  \item Create JSON-LD contexts for agent-based reasoning
  \item Develop import scripts for compatible resources
  \item Create cross-reference links to external resources
\end{enumerate}
