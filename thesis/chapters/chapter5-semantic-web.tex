\chapter{Semantic Web as Integration Interface}

\section{Motivation for Semantic Web Technologies}

The logic-witness formalism requires machine-readable specifications that:
\begin{enumerate}
\item Decouple formalism from implementation technology
\item Enable validation and reasoning about categorical structures
\item Provide a standard interchange format
\item Link to external knowledge bases (DBpedia, Wikidata)
\item Support incremental development and modular composition
\end{enumerate}

\textbf{Solution}: RDF/OWL ontologies with SHACL validation and SPARQL querying.

\section{RDF/TTL Specifications}

\subsection{Resource Description Framework (RDF)}

\textbf{RDF} provides a graph-based data model where:
\begin{itemize}
\item \textbf{Resources}: Entities (logics, theories, axioms, morphisms)
\item \textbf{Properties}: Relationships between resources
\item \textbf{Triples}: (subject, predicate, object) statements
\end{itemize}

\textbf{Turtle (TTL)}: A human-readable serialization of RDF.

\subsection{Example: Classical Logic in RDF}

\begin{lstlisting}[language=Turtle, caption=Classical Logic (LK) in TTL]
@prefix catty: <https://owner.github.io/Catty/ontology#> .
@prefix dbr: <http://dbpedia.org/resource/> .
@prefix rdfs: <http://www.w3.org/2000/01/rdf-schema#> .
@prefix skos: <http://www.w3.org/2004/02/skos/core#> .

catty:ClassicalLogic a catty:Logic ;
    rdfs:label "Classical Logic (LK)"@en ;
    catty:hasStructuralRule catty:Exchange,
                            catty:WeakeningLeft,
                            catty:WeakeningRight,
                            catty:ContractionLeft,
                            catty:ContractionRight ;
    catty:hasSequentRestriction catty:UnrestrictedAntecedent,
                                 catty:UnrestrictedSuccedent ;
    catty:provesAxiom catty:LEM, catty:LNC, catty:Explosion ;
    skos:exactMatch dbr:Classical_logic .
\end{lstlisting}

\subsection{Benefits of RDF Representation}

\begin{itemize}
\item \textbf{Modularity}: Logics, theories, and morphisms are separate resources
\item \textbf{Extensibility}: New logics can be added without modifying existing specifications
\item \textbf{Interoperability}: RDF is a W3C standard with broad tool support
\item \textbf{Linked Data}: Direct integration with DBpedia, Wikidata, OpenMath
\item \textbf{Validation}: SHACL constraints enforce categorical axioms
\item \textbf{Querying}: SPARQL enables complex queries over the knowledge graph
\end{itemize}

\section{Ontology Architecture}

\subsection{Core Ontologies}

\textbf{1. Logical Formalism Ontology} (\texttt{logical-formalism.ttl}):
\begin{itemize}
\item Classes: \texttt{Logic}, \texttt{StructuralRule}, \texttt{Sequent}, \texttt{Axiom}, \texttt{Diamond}, \texttt{Morphism}
\item Properties: \texttt{hasStructuralRule}, \texttt{hasSequentRestriction}, \texttt{provesAxiom}
\item Captures the two-dimensional lattice structure
\end{itemize}

\textbf{2. Theoretical Formalism Ontology} (\texttt{theoretical-formalism.ttl}):
\begin{itemize}
\item Classes: \texttt{Theory}, \texttt{Signature}, \texttt{TheoreticalAxiom}, \texttt{Theorem}
\item Properties: \texttt{hasLogic}, \texttt{hasSignature}, \texttt{hasAxiom}, \texttt{provesTheorem}
\item Represents theories as logic-signature-axiom triples
\end{itemize}

\textbf{3. Witness Formalism Ontology} (\texttt{witness-formalism.ttl}):
\begin{itemize}
\item Classes: \texttt{WitnessProgram}, \texttt{WitnessFunctor}, \texttt{CodeGenerator}
\item Properties: \texttt{instantiatesTheory}, \texttt{targetsLanguage}, \texttt{isQuantumSafe}
\item Represents the Curry-Howard correspondence and code generation
\end{itemize}

\textbf{4. DBpedia Mappings} (\texttt{dbpedia-mappings.ttl}):
\begin{itemize}
\item Links Catty concepts to DBpedia URIs
\item Uses \texttt{owl:sameAs}, \texttt{skos:exactMatch}, \texttt{skos:closeMatch}
\item Provides grounding for logical definitions
\end{itemize}

\subsection{Example Instance Data}

Example instance files in \texttt{ontologies/examples/}:
\begin{itemize}
\item \texttt{classical-logic.ttl}: LK specification
\item \texttt{intuitionistic-logic.ttl}: LJ specification
\item \texttt{dual-intuitionistic-logic.ttl}: LDJ specification
\item \texttt{monotonic-logic.ttl}: Monotonic logic specification
\item \texttt{linear-logic.ttl}: LL specification
\end{itemize}

\section{SHACL Validation}

\subsection{Shapes Constraint Language (SHACL)}

\textbf{SHACL} validates RDF graphs against constraints:
\begin{itemize}
\item Ensure logics have required properties (structural rules, sequent restrictions)
\item Validate categorical axioms (composition, identity, commutativity)
\item Enforce lattice order (morphisms respect structural containment)
\end{itemize}

\subsection{Example SHACL Constraint}

\begin{lstlisting}[language=Turtle, caption=SHACL Constraint for Logic Class]
@prefix catty: <https://owner.github.io/Catty/ontology#> .
@prefix sh: <http://www.w3.org/ns/shacl#> .

catty:LogicShape a sh:NodeShape ;
    sh:targetClass catty:Logic ;
    sh:property [
        sh:path catty:hasStructuralRule ;
        sh:minCount 0 ;
        sh:maxCount 5 ;
        sh:class catty:StructuralRule
    ] ;
    sh:property [
        sh:path catty:hasSequentRestriction ;
        sh:minCount 2 ;
        sh:maxCount 2 ;
        sh:class catty:SequentRestriction
    ] .
\end{lstlisting}

\textbf{Interpretation}: Every logic must have 0--5 structural rules and exactly 2 sequent restrictions (LHS and RHS).

\section{SPARQL Queries}

\subsection{Example Queries}

\textbf{Query 1: Find all quantum-safe logics}

\begin{lstlisting}[language=SPARQL]
PREFIX catty: <https://owner.github.io/Catty/ontology#>

SELECT ?logic WHERE {
    ?logic a catty:Logic .
    FILTER NOT EXISTS { ?logic catty:hasStructuralRule catty:WeakeningLeft }
    FILTER NOT EXISTS { ?logic catty:hasStructuralRule catty:WeakeningRight }
    FILTER NOT EXISTS { ?logic catty:hasStructuralRule catty:ContractionLeft }
    FILTER NOT EXISTS { ?logic catty:hasStructuralRule catty:ContractionRight }
}
\end{lstlisting}

\textbf{Query 2: Find all morphisms from intuitionistic to classical logics}

\begin{lstlisting}[language=SPARQL]
PREFIX catty: <https://owner.github.io/Catty/ontology#>

SELECT ?morphism ?source ?target WHERE {
    ?morphism a catty:Morphism ;
              catty:hasSource ?source ;
              catty:hasTarget ?target .
    ?source catty:hasSequentRestriction catty:RestrictedSuccedent .
    ?target catty:hasSequentRestriction catty:UnrestrictedSuccedent .
}
\end{lstlisting}

\section{Information Flow Architecture}

\subsection{Processing Pipeline}

\[ \text{RDF Specification} \to \text{Parser} \to \text{In-Memory Graph} \to \text{Validator} \to \text{Code Generator} \to \text{Executable Program} \]

\textbf{Step 1: RDF Specification}
\begin{itemize}
\item Logic, theory, and witness specified in TTL files
\item References to external ontologies (DBpedia, Wikidata)
\end{itemize}

\textbf{Step 2: Parser}
\begin{itemize}
\item RDFLib (Python) or Apache Jena (Java) parses TTL files
\item Constructs in-memory RDF graph
\end{itemize}

\textbf{Step 3: Validator}
\begin{itemize}
\item SHACL validation ensures categorical axioms are satisfied
\item Reports violations or confirms validity
\end{itemize}

\textbf{Step 4: Code Generator}
\begin{itemize}
\item Traverses RDF graph to extract logic-theory pairing
\item Applies witness functor to generate executable code
\item Outputs program in target language (Haskell, Rust, Q\#, etc.)
\end{itemize}

\textbf{Step 5: Executable Program}
\begin{itemize}
\item The witness: constructive implementation of the theory
\item Can be software or hardware (circuit specification)
\end{itemize}

\subsection{Tool Ecosystem}

\textbf{Python + RDFLib}:
\begin{itemize}
\item Rapid prototyping
\item RDF parsing, SPARQL queries, SHACL validation
\item Suitable for research and experimentation
\end{itemize}

\textbf{Java + Apache Jena}:
\begin{itemize}
\item Production-grade RDF processing
\item SPARQL endpoints, reasoning, ontology manipulation
\item Suitable for large-scale synthesis engine
\end{itemize}

\textbf{Hybrid Approach}:
\begin{itemize}
\item Prototype in Python (fast iteration)
\item Deploy in Java (production robustness)
\end{itemize}

\section{DBpedia Alignment}

\subsection{Strategy}

\textbf{Grounding via Linked Data}:
\begin{itemize}
\item Link Catty concepts to DBpedia URIs using \texttt{owl:sameAs}, \texttt{skos:exactMatch}
\item Leverage DBpedia's structured knowledge (extracted from Wikipedia)
\item Validate definitions against established mathematical concepts
\end{itemize}

\subsection{Example Mappings}

\begin{itemize}
\item \texttt{catty:ClassicalLogic} $\to$ \texttt{dbr:Classical\_logic} (exact match)
\item \texttt{catty:IntuitionisticLogic} $\to$ \texttt{dbr:Intuitionistic\_logic} (exact match)
\item \texttt{catty:SequentCalculus} $\to$ \texttt{dbr:Sequent\_calculus} (exact match)
\item \texttt{catty:StructuralRule} $\to$ \texttt{dbr:Structural\_rule} (exact match)
\end{itemize}

\subsection{Benefits}

\begin{itemize}
\item \textbf{Validation}: Ensures Catty definitions align with established mathematical concepts
\item \textbf{Interoperability}: Catty ontologies can be integrated with other knowledge graphs
\item \textbf{Discoverability}: Linked data enables discovery via semantic web search engines
\end{itemize}

\section{Deployment to GitHub Pages}

\subsection{Publishing Ontologies}

Ontologies are published to GitHub Pages for public access:
\begin{itemize}
\item \textbf{Base URI}: \texttt{https://owner.github.io/Catty/ontology/}
\item \textbf{Content negotiation}: Serve TTL for RDF clients, HTML for browsers
\item \textbf{Dereferenceable URIs}: All Catty concepts have resolvable URIs
\end{itemize}

\subsection{Web Interface}

GitHub Pages provides:
\begin{itemize}
\item Landing page with project overview
\item Download links for thesis PDF
\item Ontology browser (optional: generated from RDF)
\item SPARQL endpoint (optional: for advanced queries)
\end{itemize}

\section{Vacancies: Future Integration}

The following are identified for future work:
\begin{itemize}
\item Full SPARQL endpoint for querying the Catty knowledge graph
\item Interactive ontology browser (web-based visualization)
\item Automated code generation from RDF specifications (witness functor implementation)
\item Integration with external proof assistants (Isabelle, Coq, Lean)
\item Reasoning engine for inferring new morphisms or validating categorical axioms
\item Alignment with additional knowledge bases (MathWorld, Metamath, OpenMath)
\end{itemize}
