\chapter*{Morphisms --- How Logics Relate}
\addcontentsline{toc}{chapter}{Morphisms --- How Logics Relate}

\section*{Extension Morphisms}
The addition of specific axioms is generally equivalent in the sequent presentation to the addition of specific operations, combinations of operations, or combinations of operations and structural rules. This matrix or vectorization of the logical relations between units, axioms, rules, and operations is a key result that allows automatic construction of various categorizable logics in relation to each other.

This is deeply related to the form of the reflexive axiom schemata for sequent calculi. If a logic proves every reflexive axiom scheme of a given form, it will have specific structures or operations and a characterizing cut metarule. If the logic does not have the corresponding structures and operations, then either they're implicit or the logic is inconsistently characterized with respect to the reflexive axiom scheme. This is again related to the classical functional incompleteness of the subclassical logics.

\section*{Interpretation Morphisms}
Given that the initial logics are closely related to or isomorphic to logics of qubits, the embedding problems are related to the embedding of quantum logics into classical logics and the embedding of classical logic into intuitionistic or linear logic. Interpretation morphisms represent these fundamental embedding relationships and their preservation or loss of logical properties.

\section*{Functors Between Logic Categories}
Higher-order morphisms that map between different logical frameworks while preserving their internal structure, such as the Curry-Howard functor.
