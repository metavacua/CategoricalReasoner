\chapter{Introduction}

\section{Motivation}

The proliferation of formal logics---classical, intuitionistic, linear, relevance, and many others---demands a systematic framework for understanding their relationships. This white paper presents the \emph{logic-witness formalism}, a categorical approach grounded in sequent calculi and physical realizability.

Our central thesis: \textbf{Formal logics form a two-dimensional lattice category}, where:
\begin{itemize}
\item The \textbf{horizontal dimension} captures antecedent/succedent restrictions (classical $\leftrightarrow$ intuitionistic $\leftrightarrow$ dual-intuitionistic $\leftrightarrow$ monotonic)
\item The \textbf{vertical dimension} captures structural rule presence/absence (full structural $\leftrightarrow$ affine $\leftrightarrow$ relevant $\leftrightarrow$ linear $\leftrightarrow$ non-structural)
\end{itemize}

This structure is not merely organizational---it reflects deep connections between logical expressiveness, computational behavior, and physical constraints from quantum mechanics.

\section{The Two-Dimensional Structure}

\subsection{Horizontal Dimension: Sequent Restrictions}

Consider a fixed set of structural rules. Within this configuration, logics differ by how sequents are restricted:

\begin{center}
\begin{tabular}{l|l|l}
\textbf{Antecedent} & \textbf{Succedent} & \textbf{Logic} \\ \hline
Unrestricted ($\Gamma$) & Unrestricted ($\Delta$) & Classical (LK) \\
Unrestricted ($\Gamma$) & Restricted ($A$) & Intuitionistic (LJ) \\
Restricted ($A$) & Unrestricted ($\Delta$) & Dual-Intuitionistic (LDJ) \\
Restricted ($A$) & Restricted ($B$) & Monotonic
\end{tabular}
\end{center}

These four logics form a \textbf{horizontal diamond}---a commutative diagram parametrized by the presence/absence of the Law of Excluded Middle (LEM) and Law of Non-Contradiction (LNC).

\subsection{Vertical Dimension: Structural Rules}

Across logical families, structural rules vary:
\begin{itemize}
\item \textbf{Exchange (E)}: Order of premises is irrelevant
\item \textbf{Weakening (W)}: Premises can be added freely (corresponds to erasure in computation)
\item \textbf{Contraction (C)}: Duplicate premises can be merged (corresponds to copying in computation)
\end{itemize}

Five binary parameters (E, $W_L$, $W_R$, $C_L$, $C_R$) yield $2^5 = 32$ theoretical combinations. Meaningful clusters form logical families:
\begin{itemize}
\item \textbf{Full structural}: E, W, C all present (classical logic family)
\item \textbf{Affine}: E, W present; C absent (allows erasure, no cloning)
\item \textbf{Relevant}: E, C present; W absent (allows cloning, no erasure)
\item \textbf{Linear}: E present; W, C absent (quantum-safe core)
\item \textbf{Non-structural}: E, W, C all absent (maximally resource-constrained)
\end{itemize}

\subsection{The Full Category}

The complete structure is a \textbf{lattice of lattices}:
\begin{itemize}
\item Each horizontal plane represents a logical family's diamond
\item Vertical arrows represent extension morphisms respecting structural containment
\item Morphisms compose, forming a category with both 2D (horizontal) and 3D (vertical) commutative diagrams
\end{itemize}

\section{Physical Grounding}

The categorical structure is not abstract formalism for its own sake---it is grounded in \textbf{physical realizability}:

\begin{itemize}
\item \textbf{No-cloning theorem} (Wootters \& Zurek, 1982): Quantum states cannot be copied $\Rightarrow$ logics without contraction (C=0) are quantum-realizable
\item \textbf{No-erasure theorem} (Pati \& Braunstein, 2000): Quantum information cannot be destroyed $\Rightarrow$ logics without weakening (W=0) are quantum-safe
\item \textbf{Non-structural logics}: Both C=0 and W=0 $\Rightarrow$ maximally quantum-safe
\end{itemize}

This connection is precise: structural rules in proof theory correspond directly to resource management constraints in computation and physical laws in quantum mechanics.

\section{The Witness Formalism}

Through the \textbf{Curry-Howard-Kleene-Lambek correspondence}, logic, types, proofs, and programs are deeply connected:
\begin{center}
\textbf{Logic} = \textbf{Type System} \\
\textbf{Proof} = \textbf{Program} \\
\textbf{Proof Composition} = \textbf{Program Composition}
\end{center}

We extend this with the \textbf{witness functor}:
\[ W: (\text{Logic}, \text{Theory}) \to \text{ExecutableProgram} \]

A logic-theory pairing does not merely specify behavior---it \emph{instantiates} as executable code or physical circuit. The witness is constructive: it \emph{is} what the theory claims.

\section{Scope and Structure of This White Paper}

\subsection{What This Paper Provides}

\begin{enumerate}
\item \textbf{Correct abstract structure}: The two-dimensional lattice category is fully specified
\item \textbf{Logical signatures}: Formal characterization of what makes a logic categorizable
\item \textbf{Theoretical signatures}: Meta-theorems governing relationships within the category
\item \textbf{Physical grounding}: Precise connection between structural rules and quantum mechanics
\item \textbf{Robinson arithmetic case study}: Instantiation across all canonical logics
\item \textbf{Semantic web formalization}: RDF/OWL ontologies for machine-readable specifications
\end{enumerate}

\subsection{Identified Vacancies}

This white paper focuses on establishing the correct categorical structure. The following are explicitly identified as future work:

\begin{itemize}
\item Formal proofs of all meta-theorems (statements are correct; proofs are deferred)
\item Complete specification of all morphisms between logical families
\item Detailed commutative diagrams for every horizontal diamond
\item Full vertical hierarchy of all 32 structural rule combinations
\item Natural transformations and categorical limits/colimits
\item Decidability properties and their relationship to logical structure
\item Mechanized verification in Isabelle/AFP (long-term goal)
\end{itemize}

These vacancies are intentional: the foundation is correct and complete at this stage; details follow incrementally.

\subsection{Incremental Development Approach}

This repository adopts an \textbf{issue-driven development model}:
\begin{itemize}
\item Each theorem, proof, or chapter extension is tracked as a GitHub issue
\item No placeholders: sections are either substantive or explicitly marked as future work
\item Contributions follow a clear workflow (see \texttt{CONTRIBUTING.md})
\item Semantic web specifications are validated before merging
\end{itemize}

\section{Organization}

\textbf{Part I: Metamathematical Formalism}
\begin{itemize}
\item Chapter 2: Logical Formalism and the Category of Categorizable Logics
\item Chapter 3: Theoretical Formalism (Tarski-Mostowski-Robinson Framework)
\item Chapter 4: Witness Formalism (Curry-Howard-Kleene-Lambek Correspondence)
\end{itemize}

\textbf{Part II: Technical Implementation}
\begin{itemize}
\item Chapter 5: Semantic Web Integration (RDF/OWL Specifications)
\item Chapter 6: Technical Roadmap and Future Work
\end{itemize}

\textbf{Appendices}:
\begin{itemize}
\item Appendix A: External Ontologies and Vocabularies
\item Appendix B: SPARQL Query Examples
\item Appendix C: License and Attribution
\end{itemize}

\section{Prerequisites and Audience}

This white paper assumes familiarity with:
\begin{itemize}
\item \textbf{Mathematical logic}: Sequent calculi, proof theory, model theory
\item \textbf{Category theory}: Categories, functors, natural transformations, commutative diagrams
\item \textbf{Type theory}: Curry-Howard correspondence, constructive logic
\item \textbf{Semantic web}: RDF, OWL, ontologies (for implementation chapters)
\end{itemize}

The intended audience includes:
\begin{itemize}
\item Researchers in proof theory and formal logic
\item Category theorists interested in applications to logic
\item Programming language designers working on type systems
\item Quantum computing researchers interested in quantum-safe logics
\item Semantic web practitioners modeling formal knowledge
\end{itemize}
