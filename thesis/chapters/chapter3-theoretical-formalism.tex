\chapter{Theoretical Formalism}

\section{Tarski-Mostowski-Robinson Foundation}

\subsection{Theories as Structured Pairs}

A \textbf{formal theory} $T$ is defined as:
\[ T = (\text{LogicalSignature} \cup \text{TheoreticalSignature}) \otimes (\text{LogicalAxioms} \cup \text{TheoreticalAxioms}) \]

\textbf{Components}:
\begin{itemize}
\item \textbf{LogicalSignature}: Determined by choice of categorizable logic $L$
  \begin{itemize}
  \item Structural rules: $(E, W_L, W_R, C_L, C_R)$
  \item Sequent restrictions: (LHS-restricted, RHS-restricted)
  \item Logical connectives: $\{\wedge, \vee, \neg, \to, \bot, \top, \ldots\}$ (subset determined by $L$)
  \item Quantifiers: $\forall, \exists$ (if first-order)
  \end{itemize}
\item \textbf{TheoreticalSignature}: Domain-specific predicates, functions, constants
  \begin{itemize}
  \item Example (arithmetic): $\{0, S, +, \times, <, \ldots\}$
  \item Example (geometry): $\{\text{Point}, \text{Line}, \text{Between}, \ldots\}$
  \end{itemize}
\item \textbf{LogicalAxioms}: Axioms specific to logic $L$
  \begin{itemize}
  \item Explosion: $\bot \to Q$ (if $W$ present)
  \item LEM: $P \vee \neg P$ (if succedent unrestricted)
  \item LNC: $\neg(P \wedge \neg P)$ (if antecedent unrestricted)
  \end{itemize}
\item \textbf{TheoreticalAxioms}: Domain-specific axioms
  \begin{itemize}
  \item Example (Robinson): Seven axioms for arithmetic
  \item Example (Euclidean geometry): Five postulates
  \end{itemize}
\end{itemize}

\subsection{Consistency of Theories}

The consistency of $T$ depends on:
\begin{enumerate}
\item \textbf{Internal consistency of $L$}: The logic itself must be consistent
\item \textbf{Compatibility of LogicalAxioms and TheoreticalAxioms}: The two sets must not contradict each other
\end{enumerate}

\textbf{Key insight}: Changing the logic $L$ while preserving the theoretical signature may require adapting theoretical axioms to maintain consistency.

\section{Theory Transformations Under Logic Changes}

\subsection{The Transformation Problem}

\textbf{Given}: Theory $T$ in logic $L$

\textbf{Goal}: Construct theory $T'$ in logic $L'$ that ``corresponds'' to $T$

\textbf{Challenges}:
\begin{itemize}
\item Some logical axioms in $L$ may not exist in $L'$ (e.g., LEM in classical but not intuitionistic)
\item Some theoretical axioms may depend on logical properties unavailable in $L'$
\item Expressiveness may differ (e.g., negation in classical vs. monotonic)
\end{itemize}

\subsection{Systematic Transformation Procedure}

\textbf{Step 1: Identify Dependencies}
\begin{itemize}
\item Determine which theoretical axioms depend on LEM, LNC, Explosion, or other logical properties
\item Classify axioms by logical requirements
\end{itemize}

\textbf{Step 2: Rewrite or Eliminate Axioms}
\begin{itemize}
\item \textbf{Rewrite}: Adapt axioms to $L'$ if possible (e.g., constructivize for intuitionistic logic)
\item \textbf{Eliminate}: Remove axioms that cannot be expressed in $L'$ (e.g., negation-dependent axioms in monotonic)
\item \textbf{Restrict}: Limit applicability of axioms to positive fragments
\end{itemize}

\textbf{Step 3: Verify Consistency}
\begin{itemize}
\item Ensure $T'$ is consistent in $L'$
\item Check that rewritten axioms preserve intended semantics where possible
\end{itemize}

\textbf{Step 4: Generate $T'$}
\begin{itemize}
\item Construct $T' = (\text{LogicalSignature}_{L'} \cup \text{TheoreticalSignature}) \otimes (\text{LogicalAxioms}_{L'} \cup \text{TheoreticalAxioms}')$
\item Document changes and limitations
\end{itemize}

\subsection{Examples of Theory Transformations}

\subsubsection{Classical Robinson $\to$ Intuitionistic Robinson}

\textbf{Changes}:
\begin{itemize}
\item Remove LEM-dependent proofs (not axioms themselves, but derived theorems)
\item Constructivize existential statements (require explicit witnesses)
\item All seven Robinson axioms remain (they are constructively valid)
\end{itemize}

\textbf{Expressiveness}: Intuitionistic Robinson is strictly weaker (fewer theorems provable).

\subsubsection{Classical Robinson $\to$ Monotonic Robinson}

\textbf{Changes}:
\begin{itemize}
\item Eliminate Axiom 1: $\neg S(x) = 0$ (requires negation)
\item Eliminate Axiom 3: $x \neq 0 \to \exists y(x = S(y))$ (requires implication)
\item Retain Axioms 2, 4, 5, 6, 7 (positive, no negation/implication)
\end{itemize}

\textbf{Result}: Drastically limited expressiveness; monotonic Robinson is a weak fragment.

\subsubsection{Classical Robinson $\to$ Linear Robinson}

\textbf{Changes}:
\begin{itemize}
\item Adapt axioms for single-use resources
\item Modify equality to linear equality (resource-aware)
\item Rewrite successor, addition, multiplication for linear context
\end{itemize}

\textbf{Result}: Different proof strategies; resource consumption is explicit.

\section{Robinson Arithmetic as Canonical Case Study}

\subsection{Robinson's Q: Seven Axioms}

Robinson arithmetic (Q) consists of seven axioms over the signature $\{0, S, +, \times\}$:

\begin{enumerate}
\item $\neg S(x) = 0$ \quad (Zero is not a successor)
\item $S(x) = S(y) \to x = y$ \quad (Successor is injective)
\item $x \neq 0 \to \exists y(x = S(y))$ \quad (Non-zero numbers have predecessors)
\item $x + 0 = x$ \quad (Additive identity)
\item $x + S(y) = S(x + y)$ \quad (Recursive addition)
\item $x \times 0 = 0$ \quad (Multiplicative zero)
\item $x \times S(y) = x \times y + x$ \quad (Recursive multiplication)
\end{enumerate}

\subsection{Robinson in Different Logics}

\subsubsection{Classical Robinson (LK)}

\textbf{Logical Signature}: Full structural rules, unrestricted sequents, LEM, LNC, Explosion

\textbf{Axioms}: All seven axioms hold

\textbf{Theorems}: All classical arithmetic tautologies, LEM-dependent results

\textbf{Decidability}: Essentially undecidable (Robinson, 1950)

\subsubsection{Intuitionistic Robinson (LJ)}

\textbf{Logical Signature}: Full structural rules, restricted RHS, no LEM, LNC present, Explosion

\textbf{Axioms}: All seven axioms (constructively valid)

\textbf{Theorems}: Subset of classical theorems; existential claims require witnesses

\textbf{Decidability}: Essentially undecidable (constructively)

\textbf{Fragment Relationship}: $\text{theorems}_{\text{LJ}} \subset \text{theorems}_{\text{LK}}$

\subsubsection{Dual-Intuitionistic Robinson (LDJ)}

\textbf{Logical Signature}: Full structural rules, restricted LHS, LEM present, no LNC

\textbf{Axioms}: Modified axioms for dual-intuitionistic context

\textbf{Theorems}: Different subset of classical theorems; refutations require counter-witnesses

\textbf{Decidability}: Essentially undecidable (dual-intuitionistic)

\subsubsection{Monotonic Robinson}

\textbf{Logical Signature}: Full structural rules, both restricted, no LEM, no LNC, no negation/implication

\textbf{Axioms}: Axioms 2, 4, 5, 6, 7 only (Axioms 1, 3 eliminated)

\textbf{Theorems}: Drastically limited; only positive lattice operations

\textbf{Decidability}: Potentially decidable (due to severe restrictions; requires formal proof)

\subsubsection{Linear Robinson (LL)}

\textbf{Logical Signature}: Exchange, no weakening, no contraction, unrestricted sequents (resource-sensitive)

\textbf{Axioms}: All seven axioms adapted for linear context

\textbf{Theorems}: Same algorithmic structure, different resource semantics

\textbf{Decidability}: Essentially undecidable (linear logic is undecidable in general)

\subsection{Why Robinson Arithmetic?}

Robinson arithmetic serves as the canonical case study because:
\begin{enumerate}
\item \textbf{Minimal yet non-trivial}: Seven axioms capture essential arithmetic
\item \textbf{Well-studied}: Extensive literature on decidability, consistency, interpretations
\item \textbf{Expressible in all canonical logics}: All four horizontal positions (LK, LJ, LDJ, Monotonic) have Robinson variants
\item \textbf{Clear interpretation}: Successor, addition, multiplication are intuitive
\item \textbf{Undecidability}: Provides a reference point for computational complexity
\end{enumerate}

\section{Peano Arithmetic vs. Heyting Arithmetic}

\subsection{Peano Arithmetic (PA)}

\textbf{Definition}: Classical logic (LK) + Robinson's Q + Induction schema

\textbf{Induction Schema}:
\[ [P(0) \wedge \forall x (P(x) \to P(S(x)))] \to \forall x P(x) \]

\textbf{Strength}: PA is a powerful theory; proves many arithmetic truths

\textbf{Consistency}: Proven consistent relative to set theory (but not provable within PA by Gödel's Second Incompleteness Theorem)

\subsection{Heyting Arithmetic (HA)}

\textbf{Definition}: Intuitionistic logic (LJ) + Constructive Robinson + Constructive induction

\textbf{Constructive Induction}: Same schema as PA, but interpreted constructively (requires explicit witnesses)

\textbf{Strength}: HA is strictly weaker than PA (fewer provable theorems)

\textbf{Fragment Relationship}: $\text{theorems}_{\text{HA}} \subset \text{theorems}_{\text{PA}}$

\subsection{Key Insight: Logic Determines Provability}

\textbf{Same theoretical signature} (arithmetic), \textbf{different logics} $\Rightarrow$ \textbf{different theorems}.

\begin{itemize}
\item Every Heyting theorem is a Peano theorem (LJ is a fragment of LK)
\item \textbf{Converse is false}: PA has LEM-dependent theorems unprovable in HA
\end{itemize}

\textbf{Generalization}: This relationship scales to any two logics $L_1, L_2$ where $L_1$ is a fragment of $L_2$ (i.e., $L_1$ has more restrictions than $L_2$).

\subsection{Computational Interpretation}

\begin{itemize}
\item \textbf{PA proofs}: May use non-constructive methods (LEM, proof by contradiction)
\item \textbf{HA proofs}: Must be constructive (explicit witnesses for existential claims)
\end{itemize}

\textbf{Curry-Howard perspective}: HA proofs are programs; PA proofs may not be directly executable.

\section{Vacancies: Future Formalization}

The following require further development:
\begin{itemize}
\item Formal proofs of decidability properties for Robinson variants
\item Complete transformation procedures for all logic pairs
\item Mechanized verification of theory transformations
\item Decidability of monotonic Robinson (currently conjectured decidable)
\item Formal semantics for linear Robinson
\item Extension to higher-order logics and type theories
\end{itemize}
